\documentclass{beamer}

\usepackage[utf8]{inputenc}
\usepackage{ae}
\usepackage{graphicx}
\usepackage{epsfig}
\usepackage{amsmath}
\usepackage{amssymb}
\usepackage{subcaption}
\usepackage{multirow}
\usepackage{float}
\usepackage{amsthm}
\usepackage{url}         % formats URL addresses properly
\usepackage{appendix}    % allows appendix section to be included
\usepackage{lscape}      % allows a page to be rendered in landscape mode
\usepackage{multicol}    % allows text in multi columns
\usepackage{cancel}      % needed to show canceled terms in equations
\usepackage{lettrine}
\usepackage{multirow}
\usepackage{float}
\usepackage{placeins}
\usepackage{tikz}
\usetikzlibrary{shapes.geometric, arrows, positioning}

\tikzstyle{startstop} = [rectangle, rounded corners, 
minimum width=1.5cm, 
minimum height=0.5cm,
text centered, 
draw=black, 
fill=red!30]

\tikzstyle{io} = [trapezium, 
trapezium stretches=true, % A later addition
trapezium left angle=70, 
trapezium right angle=110, 
minimum width=1.5cm, 
minimum height=0.5cm, text centered, 
draw=black, fill=blue!30]

\tikzstyle{process} = [rectangle, 
minimum width=1.5cm, 
minimum height=0.5cm, 
text centered, 
text width=1.5cm, 
draw=black, 
fill=lightgray!30]

\tikzstyle{decision} = [diamond, 
% minimum width=1cm, 
% minimum height=1cm, 
text centered, 
draw=black, 
fill=green!30]
\tikzstyle{arrow} = [thick,->,>=stealth]


%HHHHHHHHHHHHHHHHHHHHHHHHHHHHHHHHHHHHHHHHHHHHHHHHHHHHHHHHHHHHHHHHHHHHHHHHHHHHHHHHHHHHHHHHHHHHHHHHHHHHHHHHHHHH
%\usepackage{subfigure}
%\usepackage{subfigmat}
%PACOTEFIGURAS_SE _ERRADO_ESXCLUIR_ACIMA
\usepackage{booktabs}
%PACOTETABELAS_SE _ERRADO_ESXCLUIR_ACIMA
%HHHHHHHHHHHHHHHHHHHHHHHHHHHHHHHHHHHHHHHHHHHHHHHHHHHHHHHHHHHHHHHHHHHHHHHHHHHHHHHHHHHHHHHHHHHHHHHHHHHHHHHHHHHH

\newcommand{\x}{\mathbf{x}}
\newcommand{\pos}{\mathbf{r}}
\newcommand{\vel}{\mathbf{v}}
\newcommand{\acc}{\mathbf{a}}
\newcommand{\uc}{\mathbf{u}}
\newcommand{\R}{\mathbb{R}}
\newcommand{\N}{\mathbb{N}}

% \addbibresource{../report/Referencias/referencias.bib}

\mode<presentation>{
    % \usetheme{Warsaw}
    \usetheme{Madrid}
    % \usetheme{Frankfurt}
    \usecolortheme{seahorse}
}

\addtobeamertemplate{frametitle}{}{
\vskip-1em
\begin{tikzpicture}[remember picture,overlay]
\node[anchor=north east,yshift=4pt] at (current page.north east) {\includegraphics[height=0.8cm]{img/ITA_logo.png}};
\end{tikzpicture}}

\title[Orbital Maneuver Optimization]{Optimal Impulsive Orbital Maneuver Synthesis Through Direct Optimization
}

\author[P. K. Puglia]{\small Pedro Kuntz Puglia\inst{1} \and Willer Santos\inst{2} \and Emilien Flayac\inst{3}}

\institute[ITA/AESP]
{
\vspace{0.5cm}
\begin{minipage}{0.5\linewidth}
  \begin{center}
    \inst{1} ITA, Student\\
    \inst{2} ITA, Professor (AESP)\\
    \inst{3} ISAE-SUPAERO, Professor (DISC)
    \vspace{1em}
    \includegraphics[height=2.5cm]{img/ITA_logo.png}
  \end{center}
\end{minipage}
}

%???
\setbeamertemplate{navigation symbols}{}

\date{}

\begin{document}

\begin{frame}
    \titlepage
\end{frame}

% Recall the outline at each section
\AtBeginSection[]
{%
\begin{frame}
  \frametitle{Plan}
  \small
  %\tableofcontents[hideothersubsections]
  %\tableofcontents[currentsubsection,hideothersubsections]
  \tableofcontents[currentsubsection]
  \normalsize
\end{frame}
}

\section{Introduction}

\begin{frame}{Context}
    \only<1-2>{
        \begin{figure}
            \centering
            \visible<1-2>{
                \begin{subfigure}{0.4\textwidth}
                    \includegraphics[width=\textwidth]{img/ISS_altitude.png}
                    \caption{ISS reboost~\cite{ISS_reboost}}
                \end{subfigure}
            }
            \visible<2>{
                \begin{subfigure}{0.4\textwidth}
                    \includegraphics[width=\textwidth]{img/impulsive_traj_from_how_many_impulses.png}
                    \caption{Impulsive transfer~\cite{how_many_impulses}}
                \end{subfigure}
            }
        \end{figure}
    }

    \only<3-4>{
        \begin{figure}
            \centering
            \visible<3-4>{
                \begin{subfigure}{0.4\textwidth}
                    \includegraphics[width=\textwidth]{img/ITASAT-2_3.png}
                    \caption{ITASAT2~\cite{itasat2_site}}
                \end{subfigure}
            }
            \visible<4>{
                \begin{subfigure}{0.4\textwidth}
                    \includegraphics[width=\textwidth]{img/Amazonia1.jpeg}
                    \caption{Amazonia 1 PMM~\cite{amazonia1_inpe}}
                \end{subfigure}
            }
        \end{figure}
    }
\end{frame}

\begin{frame}{Problem Statement}
    \begin{block}{Central Question}
        What is the most efficient sequence of impulses that takes a spacecraft from an initial state to a final state in a given time?
    \end{block}
    
    \begin{itemize}
        \pause
        \item LEO environment: applicability \& perturbations
        \item General case in mind (no particular analytical solutions) 
        \item Compare maneuvers with different perturbation models
        \item Local solutions backed by optimal control theory: \textbf{primer vector}
        \begin{itemize}
            \item How many impulses?
            \item Develop theory for perturbations
        \end{itemize}
    \end{itemize}
\end{frame}

% \begin{frame}
%     \frametitle{Hypotheses}

%     \begin{itemize}
%         \item Parameter optimization problem
%         \item Need fixed time for practical solution~\cite{impulsive_europa}
%         \item More impulses, revolutions \(\rightarrow\) harder problem
%         \item 
%     \end{itemize}
% \end{frame}

% \begin{frame}
%     \frametitle{Objectives}
%     \textbf{Describe a method capable of finding the sequence
%     of impulses that transfer a satellite between given orbits}. Secondary:
%     \pause
%     \begin{itemize}
%         \item Apply primer vector theory;
%         \item Study how much time fo transfer, and how to find it;
%         \item Compare numerical and analytical results;
%         \item Discuss applications in common scenarios;
%     \end{itemize}

% \end{frame}

% \begin{frame}
%     \frametitle{Justification}

%     Some institutions already know how to optimize orbital maneuvers. Why study it again?
%     \pause
%     \begin{figure}[htbp]
%         \centering
%         \begin{subfigure}{0.25\textwidth}
%             \includegraphics[width=\textwidth]{img/orekit-logo.png}
%             \caption{Orekit library provides maneuver analysis, and indirect optimization (outdated).}
%         \end{subfigure} \pause
%         \begin{subfigure}{0.25\textwidth}
%             \includegraphics[width=\textwidth]{img/patrius_logo.png}
%             \caption{CNES Patrius library provides analysis, not Synthesis.}
%         \end{subfigure} \pause
%         \begin{subfigure}{0.25\textwidth}
%             \centering
%             \includegraphics[width=\textwidth]{img/mystic_no_download.png}
%             \caption{NASA's Mystic software (Dawn mission) is not available for download.}
%         \end{subfigure}
%     \end{figure}
%     \pause
%     \textbf{No widely available orbital maneuver optimization software.}
% \end{frame}

\section{Theory}

\begin{frame}
    \frametitle{Orbital mechanics}

    \begin{block}{Two-Body Motion}
        \visible<1->{
            Keplerian dynamics:
            \begin{equation}
                \ddot{\mathbf{r}} = - \frac{\mu}{\lVert \mathbf{r} \rVert^3} \mathbf{r}
            \end{equation}
            Closed, planar, periodic elliptical trajectory for negative energy (bound satellite).
        }
    \end{block}
    \begin{columns}[t]
        \begin{column}{0.5\linewidth}
            \begin{center}
                \textbf{J2 perturbation}
            \end{center}

            \begin{itemize}
                \item Largest LEO perturbation~\cite{curtis2015orbital}
                \item Earth's oblateness
                \item Conservative force
                \begin{itemize}
                    \item \(\ddot{\pos} = g(\pos)\)
                \end{itemize}
                \item Nonplanar trajectory
            \end{itemize}
        \end{column}
        \begin{column}{0.5\linewidth}
            \begin{center}
                \textbf{Drag perturbation}
            \end{center}

            \begin{itemize}
                \item Second largest LEO perturbation
                \item Needs atmospheric model
                \item Dissipative force
                \begin{itemize}
                    \item \(\ddot{\pos} = g(\pos) + d(\pos, \vel)\)
                \end{itemize}
            \end{itemize}
        \end{column}
    \end{columns}
\end{frame}

\begin{frame}
    \frametitle{Optimal control}\pause
    \begin{block}{Generic optimal control problem}
        Given a dynamical system \(\dot{\mathbf{x}} = f(\mathbf{x}, \mathbf{u})\), a fixed initial condition \(\mathbf{x}(0) = \mathbf{x}_i\), a total time \(t_f\) and a final condition \(\mathbf{x}(t_f) = \mathbf{x}_f\), find control trajectory \(\mathbf{u}(t)\) minimizing objective \(J[\mathbf{x}(t), \mathbf{u}(t)] = h(\mathbf{x}(t_f))+\int_0^{t_f} L(x(t), u(t)) dt\). 
    \end{block}\pause
    \begin{columns}[t]
        \begin{column}{0.6\linewidth}
            \begin{center}\textbf{Indirect method}\end{center}
            \begin{itemize}
                \item Hamiltonian with Pontryagin's Minimum Principle~\cite{bertsekas}: \(\)
                \begin{align}
                    H &= L(\mathbf{x}, \mathbf{u}) + \boldsymbol{\lambda}^T f(\mathbf{x}, \mathbf{u}) \\
                    \dot{\boldsymbol{\lambda}} &= - \left( \frac{\partial H}{\partial \mathbf{x}} \right)^T \\
                    \mathbf{u}^\star(t) &\in \arg \min_{\mathbf{u} \in \mathcal{U}} H[\mathbf{x}(t), \mathbf{u}, \boldsymbol{\lambda}(t)]
                \end{align}
            \end{itemize}
        \end{column}
        \begin{column}{0.3\linewidth}\pause
            \begin{center}
                \textbf{Direct method}
            \end{center}
            \begin{itemize}
                \item Multiple shooting
                \item Numerical integration
                \item Trajectories \(\rightarrow \mathbf{x}_k, \mathbf{u}_k\)
                \item Solve for \(\mathbf{x}\), \(\mathbf{u}\)
            \end{itemize}
        \end{column}
    \end{columns}
\end{frame}

\begin{frame}
    \frametitle{OC Solution Hierarchy}

    TODO

\end{frame}

\begin{frame}
    \frametitle{Orbital Maneuvering}
    \pause
    \begin{block}{Extended dynamics}
        \begin{equation}
            \ddot{\mathbf{r}} = g(\pos) + \boldsymbol{\Gamma}.
        \end{equation}\pause
        Control input is acceleration \(\boldsymbol{\Gamma} = \Gamma \hat{\uc}\). Objective: \(\min \int_{0}^{t_f} \Gamma(t) dt\)
    \end{block}
    \pause
    \begin{itemize}
        \item Impulsive thrust
        \begin{itemize}
            \item Engineering: independent of propulsive plant
            \item Physics: high thrust \(\approx\) impulse
            \item Mathematics: minimum is unattainable with bounded acceleration
        \end{itemize}
        \item Velocity discontinuities, \textit{coasting arcs}
        \begin{equation}
            \mathbf{v}(t_i^+) = \mathbf{v}(t_i^-) + \Delta \mathbf{v}
        \end{equation}
        \item Hamiltonian results: \textit{primer vector} theory
    \end{itemize}
\end{frame}

\begin{frame}
    \frametitle{Primer vector theory}
    \begin{itemize}
        \item Hamiltonian~\cite{Conway_2010}
        \begin{equation}
            H = (1 + \boldsymbol{\lambda}_v^T \hat{\mathbf{u}}) \Gamma + \dots
        \end{equation}
        \item \((1 + \boldsymbol{\lambda}_v^T \hat{\mathbf{u}})\) as small as possible when \(\Gamma > 0\)
        \item \(\therefore \hat{\uc} \parallel - \boldsymbol{\lambda}_v\)
        \item Define primer vector \(\mathbf{p} = - \boldsymbol{\lambda}_v\): optimal direction of impulse
        \item Necessary conditions for optimal \(\boldsymbol{\Gamma}^\star\):
        \begin{itemize}
            \item \(\mathbf{p}(t)\) and \(\dot{\mathbf{p}}(t)\) are continuous;
            \item \(\lVert \mathbf{p} \rVert \leq 1\);
            \item \(\lVert \mathbf{p} \rVert = 1\) at the impulse instants (\(\mathbf{p} = \hat{\uc}\));
        \end{itemize}
        \item Suboptimal trajectory:
        \begin{itemize}
            \item (\(\lVert \mathbf{p}(0) \rVert = 1\)) and \(\partial_t \lVert \mathbf{p} \rVert \mid_{t=0} > 0\), adding an initial coast will lower the cost;
            \item \(\lVert \mathbf{p}(t) \rVert > 1\) for some \(t \in [0, t_f]\), adding an impulse will lower the cost.
        \end{itemize}
    \end{itemize}
\end{frame}


\begin{frame}
    \frametitle{Primer Vector Calculation}

    \begin{itemize}
        \item Conservative model
        \begin{equation}
            \begin{bmatrix}
                \dot{\mathbf{p}} \\ \ddot{\mathbf{p}}
            \end{bmatrix} = \begin{bmatrix}
                \mathbf{0}_3 & \mathbf{I}_3 \\
                \left[\frac{\partial g}{\partial \pos}\right]^T & \mathbf{0}_3
            \end{bmatrix} \begin{bmatrix}
                \mathbf{p} \\ \dot{\mathbf{p}}
            \end{bmatrix} = \mathbf{A}_p(\pos) \begin{bmatrix}
                \mathbf{p} \\ \dot{\mathbf{p}}
            \end{bmatrix}
        \end{equation}
        \begin{itemize}
            \item \textbf{ODE method}: numerical integration
            \item Linear ODE: Primer vector transition matrix (PVTM) \(\Phi_p(t)\)
            \item Same as ODE for variational perturbations: use state transition matrix (STM) \(\Phi_\delta(t)\)
            \item \textbf{STM method}: \(\Phi_p(t) = \Phi_\delta(t) = \frac{\partial \x(t)}{\partial \x_0}\)
            \item Keplerian model:
            \begin{itemize}
                \item \textbf{Glandorf method}: analytical form of \(\Phi_p(t) = \Phi_\delta(t)\)
            \end{itemize}
            \item \textbf{STM}, \textbf{Glandorf} widely used in literature~\cite{impulsive_europa}\cite{bruno_asteroid_primer}\cite{Conway_2010}
        \end{itemize}
    \end{itemize}

\end{frame}

\begin{frame}
    \frametitle{Primer Vector Calculation}

    \begin{itemize}
        \item Non-conservative model
        \begin{equation}\label{eq:pdot_ncon}
            \begin{bmatrix}
                \dot{\mathbf{p}} \\ \ddot{\mathbf{p}}
            \end{bmatrix} = \begin{bmatrix}
                \mathbf{0}_3 & \mathbf{I}_3 \\
                \left[\frac{\partial }{\partial \mathbf{r}}\big(g(\mathbf{r}) + d(\pos, \vel)\big)\right]^T & -\left(\frac{\partial d(\pos, \vel)}{\partial \vel}\right)^T
            \end{bmatrix} \begin{bmatrix}
                \mathbf{p} \\ \dot{\mathbf{p}}
            \end{bmatrix} = \mathbf{A}_p(\pos, \vel) \begin{bmatrix}
                \mathbf{p} \\ \dot{\mathbf{p}}
            \end{bmatrix}.
        \end{equation}
        \begin{itemize}
            \item Not found in the literature
            \item \textbf{Not} the same as variational perturbation ODE
            \item Only \textbf{ODE method} available
        \end{itemize}
    \end{itemize}

    \begin{block}{Model and primer vector summary}
        \begin{table}[htbp]
            \centering
            \begin{tabular}{ccccc} \toprule
                      &       & \multicolumn{3}{c}{\textbf{Primer vector method}} \\
                Model & Class & Glandorf & STM & ODE \\ \midrule
                Two-Body & Keplerian & \checkmark & \checkmark & \checkmark \\
                J2 & Conservative & \(\times\) & \checkmark & \checkmark \\
                J2+Drag &Non-conservative & \(\times\) & \(\times\) & \checkmark \\ \bottomrule
            \end{tabular}
            % \caption{Summary of primer vector calculation methods for different model types.}
            % \label{tab:pv_calc}
        \end{table}
    \end{block}

\end{frame}


\section{Methodology}

\begin{frame}
    \frametitle{Numerical Tools}

    \begin{itemize}
        \item<2-> Julia~\cite{Julia-2017} language
        \item<3-> \texttt{SatelliteToolbox}~\cite{satellitetoolbox}: auxiliary orbit propagation
        \item<4-> CasADi~\cite{casadi}: optimization interface
        \item<5-> Ipopt~\cite{ipopt}: robust nonlinear optimizer. Local, deterministic, gradient-based
    \end{itemize}
    
    \begin{minipage}[c][][c]{0.5\textwidth}
        \begin{center}
            \includegraphics[scale=0.2]{img/ipopt_logo.png} 
        \end{center}
    \end{minipage}\begin{minipage}[c][][c]{0.5\textwidth}
        \begin{center}
            \includegraphics[scale=0.05]{img/satellitetoolbox_logo.png}
        \end{center}
    \end{minipage}

    \begin{minipage}[c][][c]{0.5\textwidth}
        \begin{center}
            \includegraphics[scale=0.2]{img/casadi.png}
        \end{center}
    \end{minipage}\begin{minipage}[c][][c]{0.5\textwidth}
        \begin{center}
            \includegraphics[scale=0.2]{img/julia.png} 
        \end{center}
    \end{minipage}
\end{frame}

\begin{frame}
    \frametitle{Algorithms}

    \begin{itemize}\pause
        \item Maneuver representation
        \begin{itemize}
            \item \texttt{C}: coasting arc
            \item \texttt{I}: impulse
            \item \texttt{ICI}, \texttt{CICIC}, \texttt{CICICIC}, etc
        \end{itemize}
        \item Impulse: boundary condition between coasting arcs
        \item Coasting arc
        \begin{itemize}
            \item Optimization: RK8 Cartesian multiple shooting
            \item Propagation: integration or \texttt{SatelliteToolbox}
        \end{itemize}
        \item Primer vector calculation
        \begin{itemize}
            \item Solve two point boundary value problem between consecutive impulses with PVTM
        \end{itemize}
        \item Search for \textbf{number of impulses}~\cite{interactive_primer_vector}
        \begin{enumerate}
            \item Solve \texttt{ICI}
            \item Solve \texttt{CICIC}
            \item Primer vector calculation
            \item Add impulses when \(\max \lVert \mathbf{p} \rVert > 1\)
            \item Repeat 3-5 until all necessary conditions satisfied
        \end{enumerate}
    \end{itemize}
\end{frame}

\section{Results}

%scenarios
%summary with n imp & delta v
%show some res

\begin{frame}
    \frametitle{Maneuver Scenarios}

    % C2C SETUP
\begin{table}[htbp]
    \centering
    \begin{tabular}{ccccc} \toprule
                   & \multicolumn{2}{c}{Circle to Circle} & \multicolumn{2}{c}{Noncoplanar rendez-vous} \\ \midrule
        Element    & Initial        & Final              & Initial & Final \\ \midrule
        \(a\)      & \(7000.0\) km  & \(9000.0\) km      & \(6748.1\) km         & \(6778.1\) km   \\
        \(e\)      & \(0.0\)        & \(0.0\)            & \(0.0\)            & \(0.0\)        \\
        \(i\)      & \(51.0^\circ\) & \(51.0^\circ\)     & \(42.1^\circ\)      & \(42.0^\circ\) \\
        \(\Omega\) & \(0.0^\circ\)  & \(0.0^\circ\)      & \(120.2^\circ\)   & \(120.0^\circ\)  \\
        \(\omega\) & \(0.0^\circ\)  & \(0.0^\circ\)      & \(0.0^\circ\)  & \(0.0^\circ\)  \\
        \(\theta\) & \(0.0^\circ\)  & \(180.0^\circ\)    & \(175.0^\circ\)  & \(180.0^\circ\)  \\ 
        Maneuver time & \multicolumn{2}{c}{\(3560.541\)} & \multicolumn{2}{c}{\(11107.158\)} \\\bottomrule
    \end{tabular}
    % \caption{Orbital elements used for the Circle to circle transfer case analysis}
    % \label{tab:c2c_orb_elems}
    \begin{itemize}
        \item Circle to circle
        \begin{itemize}
            \item Keplerian model: Hohmann transfer
        \end{itemize}
        \item Noncoplanar rendez-vous
        \begin{itemize}
            \item Keplerian model: results in the literature~\cite{interactive_primer_vector}
        \end{itemize}
    \end{itemize}
\end{table}
% C2C SETUP

\end{frame}

\begin{frame}
    \frametitle{Keplerian Noncoplanar Rendez-vous}

    \begin{columns}
        \begin{column}{0.49\linewidth}
            \begin{figure}[htbp]
                \centering
                \includegraphics[width=\linewidth]{../results/two_body/ipv_noncop/ICI_primer_vector.png}
                % \caption{<caption>}
                % \label{<label>}
            \end{figure}
            Local Extremum
        \end{column}
        \begin{column}{0.49\linewidth}
            \begin{figure}[htbp]
                \centering
                \includegraphics[width=\linewidth]{../results/two_body/ipv_noncop/ICI_3d.png}
                % \caption{<caption>}
                % \label{<label>}
            \end{figure}
        \end{column}
    \end{columns}

    \begin{table}[]
        \small
        \begin{tabular}{cccc} \toprule
        \textbf{Impulse}   & 1           & 2           & \textbf{Total} \\ \midrule
        \(t\) (s)          & 0.0         & 11107.1576  & 11107.1576     \\ 
        \(\Delta v\) (m/s) & 11740.94035 & 11708.69678 & 23449.63713    \\ \bottomrule
        \end{tabular}
    \end{table}
\end{frame}

\begin{frame}
    \frametitle{Keplerian Noncoplanar Rendez-vous}

    \begin{columns}
        \begin{column}{0.49\linewidth}
                \begin{figure}[htbp]
                    \centering
                    \includegraphics[width=\linewidth]{../results/two_body/ipv_noncop/CICIC_primer_vector.png}
                    % \caption{<caption>}
                    % \label{<label>}
                \end{figure}
                Addition of impulse
        \end{column}
        \begin{column}{0.49\linewidth}
            \begin{figure}[htbp]
                \centering
                \includegraphics[width=\linewidth]{../results/two_body/ipv_noncop/CICIC_z+.png}
                % \caption{<caption>}
                % \label{<label>}
            \end{figure}
        \end{column}
    \end{columns}

    \begin{table}[]\small
        \begin{tabular}{cccc} \toprule
        \textbf{Impulse}   & 1          & 2           & \textbf{Total} \\ \midrule
        \(t\) (s)          & 6644.30733 & 10689.86179 & 11107.1576     \\
        \(\Delta v\) (m/s) & 37.29252   & 16.20984    & 53.50237      \\ \bottomrule
        \end{tabular}
        \end{table}
\end{frame}

\begin{frame}
    \frametitle{Keplerian Noncoplanar Rendez-vous}

    \begin{columns}
        \begin{column}{0.49\linewidth}
                \begin{figure}[htbp]
                    \centering
                    \includegraphics[width=\linewidth]{../results/two_body/ipv_noncop/CICICIC_primer_vector.png}
                    % \caption{<caption>}
                    % \label{<label>}
                \end{figure}
                Addition of impulse
        \end{column}
        \begin{column}{0.49\linewidth}
            \begin{figure}[htbp]
                \centering
                \includegraphics[width=\linewidth]{../results/two_body/ipv_noncop/CICICIC_z+.png}
                % \caption{<caption>}
                % \label{<label>}
            \end{figure}
        \end{column}
    \end{columns}

    \begin{table}[]\small
        \begin{tabular}{ccccc} \toprule
        \textbf{Impulse}   & 1         & 2          & 3          & \textbf{Total} \\ \midrule
        \(t\) (s)          & 3370.1071 & 6774.61652 & 9176.42663 & 11107.1576     \\
        \(\Delta v\) (m/s) & 11.76294  & 10.56494   & 20.74554   & 43.07342      \\ \midrule
        \end{tabular}
        \end{table}
\end{frame}

\begin{frame}
    \frametitle{Keplerian Noncoplanar Rendez-vous}

    \begin{columns}
        \begin{column}{0.49\linewidth}
                \begin{figure}[htbp]
                    \centering
                    \includegraphics[width=\linewidth]{../results/two_body/ipv_noncop/CICICICIC_primer_vector.png}
                    % \caption{<caption>}
                    % \label{<label>}
                \end{figure}
                Local extremum
        \end{column}
        \begin{column}{0.49\linewidth}
            \begin{figure}[htbp]
                \centering
                \includegraphics[width=\linewidth]{../results/two_body/ipv_noncop/CICICICIC_z+.png}
                % \caption{<caption>}
                % \label{<label>}
            \end{figure}
        \end{column}
    \end{columns}

    \begin{table}[] \small
        \begin{tabular}{cccccc} \toprule
        \textbf{Impulse}   & 1       & 2          & 3          & 4           & \textbf{Total} \\ \midrule
        \(t\) (s)          & 0.00394 & 6724.60052 & 9217.50949 & 11107.15747 & 11107.1576     \\
        \(\Delta v\) (m/s) & 3.58823 & 9.34687    & 15.01486   & 8.19599     & 36.14596      \\ \bottomrule
        \end{tabular}
        \end{table}
\end{frame}

\begin{frame}
    \frametitle{J2+Drag circle to circle}

    

\end{frame}

\begin{frame}
    \frametitle{Model Comparison}

    \begin{table}[htbp]\small
        \centering
        \begin{tabular}{ccccccc} \toprule
            Scenario                & \multicolumn{2}{c}{Keplerian} & \multicolumn{2}{c}{J2}       & \multicolumn{2}{c}{J2+Drag} \\
                                    & \(n_i\) & \(\Delta v\) (m/s)  & \(n_i\) & \(\Delta v\) (m/s) & \(n_i\) & \(\Delta v\) (m/s)\\ \midrule
            Circle to circle        & 2 & 887.56199 & 3 & 893.05336 & 3 & 893.05339 \\
            Noncoplanar rendez-vous & 4 & 36.14596  & 3 & 56.00653  & 3 & 56.01418 \\ \bottomrule
        \end{tabular}
        % \caption{<caption>}
        % \label{<label>}
    \end{table}
    \begin{itemize}
        \item Perturbations alter cost, and \textbf{number of impulses} as well
        \begin{itemize}
            \item Planar Keplerian \(\rightarrow\) nonplanar J2(+Drag)
        \end{itemize}
        \item Primer vector algorithm can offer >100x increase in performance
        \item Short time frame: drag has little effect
        \item Conjecture: 
        
        J2+Drag maneuver \(\neq\) J2 maneuver \(\iff\) STM \(\neq\) ODE method?
        \begin{itemize}
            \item Long timespans
            \item High drag: aerobraking, Very Low Earth Orbit
        \end{itemize}
    \end{itemize}
\end{frame}

\begin{frame}
    \frametitle{Conclusions}

    

\end{frame}

\begin{frame}[allowframebreaks]{References}
    \bibliographystyle{plain}
    \bibliography{Referencias/referencias}
\end{frame}

\end{document}