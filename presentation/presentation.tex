\documentclass{beamer}

\usepackage[utf8]{inputenc}
\usepackage[english]{babel}
% \usepackage{palatino}
% \usepackage{graphicx}
% \graphicspath{{./images/}}
% \usepackage{colortbl}
% \usepackage{xcolor}
\usepackage{tikz}
% \usetikzlibrary{shapes,arrows}
% \usetikzlibrary{mindmap,trees}
% \usetikzlibrary{calc}
% \usepackage{pgfplots}
% \pgfplotsset{compat=newest}
% \pgfplotsset{plot coordinates/math parser=false}
% \newlength\figureheight
% \newlength\figurewidth
% \usepackage{ifthen}
% \usepackage{subfigure}
% \usepackage{amsthm}
% \usepackage{amsfonts}
% \usepackage{amssymb}
% \usepackage{amsmath}
% \usepackage{eurosym}
% \usepackage{wasysym}

% \addbibresource{../report/Referencias/referencias.bib}

\mode<presentation>{
    % \usetheme{Warsaw}
    \usetheme{Madrid}
    % \usetheme{Frankfurt}
    \usecolortheme{seahorse}
}

\addtobeamertemplate{frametitle}{}{
\vskip-1em
\begin{tikzpicture}[remember picture,overlay]
\node[anchor=north east,yshift=4pt] at (current page.north east) {\includegraphics[height=0.8cm]{img/ITA_logo.png}};
\end{tikzpicture}}

\title[Orbital Maneuver Optimization]{Optimal Impulsive Orbital Maneuver Synthesis Through Direct Optimization
}

\author[P. K. Puglia]{\small Pedro Kuntz Puglia\inst{1} \and Willer Santos\inst{2} \and Emilien Flayac\inst{3}}

\institute[ITA/AESP]
{
\vspace{0.5cm}
\begin{minipage}{0.5\linewidth}
  \begin{center}
    \inst{1} ITA, Student\\
    \inst{2} ITA, Professor (AESP)\\
    \inst{3} ISAE-SUPAERO, Professor (DISC)
    \vspace{1em}
    \includegraphics[height=2.5cm]{img/ITA_logo.png}
  \end{center}
\end{minipage}
}

%???
\setbeamertemplate{navigation symbols}{}

\date{}

\begin{document}

\begin{frame}
    \titlepage
\end{frame}

% Recall the outline at each section
\AtBeginSection[]
{%
\begin{frame}
  \frametitle{Plan}
  \small
  %\tableofcontents[hideothersubsections]
  %\tableofcontents[currentsubsection,hideothersubsections]
  \tableofcontents[currentsubsection]
  \normalsize
\end{frame}
}

\section{Introduction}

\begin{frame}{Context}
    
\end{frame}

\begin{frame}{Problem Statement}
    \begin{block}{Central Question}
        What is the most efficient sequence of maneuvers that takes a spacecraft from an initial state to a final state in a given time?
    \end{block}
    
    \begin{itemize}
        \item Efficient: least propellant usage
        \item General case in mind (no particular analytical solutions)
        \item How much time? Feasibility, trade-offs?
        \item How many impulses?
        \item Is it optimal?
    \end{itemize}
\end{frame}

\begin{frame}
    \frametitle{Hypotheses}

    \begin{itemize}
        \item Choice for \textit{impulsive propulsion} \(\rightarrow\) reducible to parameter optimization
        \item Good numerical solvers: Ipopt\cite{ipopt}
    \end{itemize}

\end{frame}

\begin{frame}[allowframebreaks]{References}
    \bibliographystyle{plain}
    \bibliography{Referencias/referencias}
\end{frame}

\end{document}