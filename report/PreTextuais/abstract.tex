This work sets out to apply primer vector theory to optimal impulse maneuvers in LEO and adapt it to the types of perturbations encountered in this environment, which is not readily available in the literature. A review of the theory of optimal control and orbital maneuvering is made and in particular, primer vector theory is laid out in detail, including its extension to conservative and non-conservative perturbation models. An impulsive multiple shooting optimization scheme in Cartesian coordinates is presented, implemented in Julia language with the CasADi package, and validated under Keplerian dynamics for the well known Hohmann transfer case, and for a more complex noncoplanar rendez-vous scenario proposed in the primer vector literature. Then, the same scenarios are solved again under perturbed orbital dynamics with the help of the primer vector, and the resulting trajectories are compared. The perturbation models include a J2 model, representing the class of conservative perturbations by modelling Earth's nonspherical gravity field, and a J2+Drag model, representing the class of non-conservative perturbations by including the effects of atmospheric drag in LEO\@. For each model, some valid methods of primer vector calculation have been tried and validated between each other, and the primer vector is proven to be a useful tool in reducing the cost of orbital maneuvers.