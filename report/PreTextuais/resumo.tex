Este trabalho busca aplicar a teoria do vetor primário a manobras impulsivas ótimas em órbita baixa terrestre (LEO) e adaptá-la aos tipos de perturbações encontrados neste ambiente, o que não está facilmente disponível na literatura. Uma revisão da teoria de controle ótimo e manobras orbitais é feita e em particular, a teoria do vetor primário é apresentada em detalhes, incluindo sua extensão para modelos de perturbações conservativas e não-conservativas. Um esquema de otimização impulsive baseado em \textit{multiple shooting} é apresentado, implementado em Julia com a biblioteca CasADi, e validada sob dinâmica Kepleriana para o conhecido caso da transferência de Hohmann e um caso mais complexo de \textit{rendez-vous} não-coplanar, proposto na literatura de primer vector. Em seguida, os mesmos cenários são resolvidos novamente sob dinâmica orbital perturbada com a ajuda do vetor primário, e as trajetórias resultantes são comparadas. Os modelos perturbatórios incluem um modelo J2, representando a classe de perturbações conservativas através da modelagem da gravidade não-esférica da Terra, e um modelo J2+Arrasto, representando a classe de perturbações não-conservativas através da inclusão de efeitos do arrasto atmosférico em LEO\@. Para cada modelo, alguns métodos de cálculo do vetor primário foram testados e validados entre si, e o vetor primário provou-se uma ferramenta útil na redução do custo de manobras orbitais.