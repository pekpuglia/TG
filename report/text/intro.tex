

% primeiro satelite manobravel


% discutir manobra interplanetária vs órbita terrestre

% GOCE

% daedalus

Space exploration relies on clever resource management, since satellites have a finite amount of resources (propellant and other consumables) to fulfill their mission. Up to this date, all space hardware is expendable, that is, when the consumables required for mission maintenance are finished, the mission ends, marking the end of the exploitation of a very expensive engineered system. The main consumable aboard a satellite is \textit{propellant}, which is used in the satellite's thrusters to impart changes in its trajectory. Changing a satellite's trajectory requires an \textit{orbital maneuver}, that is, a series of thruster activations that removes the satellite from its initial, natural trajectory and takes it to a final, more desirable orbit. There are many possible maneuvers that lead to the same final orbit, with varying propellant consumptions. Finding the smallest propellant consumption is therefore of great interest, which makes the orbital maneuvering problem an \textit{optimal control} problem.

In practice, the nature of a maneuver greatly depends on the propulsion system onboard. Contrary to science fiction, where spaceships seem to be constantly propelled by their thrusters, real life satellites with chemical thrusters change their courses in discrete moments of maximum thrust application, surrounded by (usually long) coasting periods. This is due to the relatively high power delivered by traditional rocket engines, which can, in the matter of seconds or minutes, greatly alter a satelite's orbit. These moments of propelled flight can, to good approximation, be considered as punctual discontinuities in the satellite's velocity. They are called \textit{impulsive maneuvers}. Certain more modern propulsion systems, such as electric rocket engines, are an exception in that the thrust they provide is quite low, and they operate for extended periods of time. In this case, the propelled arcs are long, and the satellite's motion during propulsion must be taken into account. Despite being more modern, electric propulsion systems are not ubiquitous, and chemical propulsion remains the best solution for many applications. 

Impulsive maneuvers are described by the impulse magnitudes and the number of impulses. From the point of view of optimization, handling the number of impulses is a challenge, since it is a discrete parameter that cannot be treated as a variable in standard optimal control tools. In practice, this is a problem: the difference in cost of a 2-impulse maneuver and a 4-impulse maneuver can be dramatic, exceeding for instance the cost to escape Earth's gravity field, as will be seen in the Results section. This means that this is not just a theoretical problem, but quite a pratical one. Luckily, the theory of optimal control applied to orbital maneuvering offers a tool for helping in the search of the number of impulses: the \textit{primer vector}~\cite{Conway_2010}.

The reasons for performing an orbital maneuver are many: placing the satellite in the correct orbit at the beginning of its lifetime, placing the satellite in a \textit{graveyard} orbit at the end of lifetime, or applying small corrections along its mission to prevent it from straying too far from the mission's requirements. \textit{Maintenance maneuvers} mitigate external perturbations such as atmospheric drag, Earth's oblateness effects (if undesired), gravitational attraction of celestial bodies, and solar radiation pressure during the mission's lifetime to keep the satellite close to the desired orbit. Their frequency and magnitude vary depending on mission requirements and, in industrial applications, other mission requirements must be taken into account when planning maneuvers. The presence of sensitive sensors that must not be pointed at the Sun, solar panels that must always be illuminated, or events such as observation of a ground target are examples of sources of constraints on when maneuvers can be executed. Those are by far the most common type of maneuver, and a loose, non-exhaustive classification arises naturally.

The simplest type of maneuver is that of \textit{orbit raising}, which consists in taking the satellite from a (often near-circular) orbit and increasing its semimajor-axis (and thus, its period) until a desired value. This maneuver is commonly found in Low Earth Orbit (LEO) applications, due to the presence of atmospheric drag; notably, it is performed by the Internacional Space Station (ISS) about once a month~\cite{iss_reboost}\@. From a theoretical standpoint, it presents a simple, introductory case, often restricted to two dimensions instead of three. There are plenty of theoretical results about it, most notably the Hohmann transfer~\cite{chobotov}, a two-impulse maneuver which is known to be the two-impulse optimal from a plethora of theoretical tools. Other more elaborate results include the bielliptic transfer~\cite{chobotov}, which can be shown to surpass Hohmann's performance in certain conditions by allowing a third impulse.

A second type of maneuver is a \textit{plane change} maneuver~\cite{curtis2015orbital}. Satellites move (approximately) in a plane which contains its position and velocity vectors and the center of Earth. By changing the direction of the velocity, this plane may be change. Common cases include an inclination change during orbital insertion, which may be required if the inclination of the target orbit is different to the latitude of the launch center~\cite{curtis2015orbital}. Another type of maneuver is the \textit{phasing} maneuver~\cite{curtis2015orbital}. This maneuver consists in changing the position occupied by the satellite within the same orbit at a certain  time. This maneuver is very important for \textit{orbital rendez-vous}, where not only it is required that two vessels share the same orbit, but also they must have the same position and velocity at the same time. A notable, recurring example of rendez-vous is that between the Soyuz capsule and the ISS, which requires agreement of all orbital elements and the correct phasing. This is usually a multi-revolution maneuver but recent advances have greatly reduced the time required for the rendez-vous~\cite{soyuz_iss}. Finally, at the end-of-life, there are legal constraints on where a satelite may be disposed of. NASA LEO missions have a deadline of 25 years for deorbiting into Earth's atmosphere~\cite{nasa_deorbit}, while GEO satellites are usually placed into a cemetery orbit which does not intersect the highly prized GEO region. As an end-of-life procedure, feasibility is of utmost importance, while ensuring optimality increases the lifespan of the mission.

Of particular interest to Brazil's space program are maneuvers in LEO\@. The ITASAT-2 project proposes a flight formation of 3 satellites in LEO that must keep precise positions through the usage of chemical thrusters~\cite{itasat2_conops}. While previous work considers relative position control in a small deviations context~\cite{itasat2}, this work shall tackle the full problem of planning generic orbital maneuvers in LEO. 


\section{Problem statement}

The central question of this work is how to find the most efficient sequence of impulsive maneuvers that take a satellite from an initial state to a final state in a given amount of time, under perturbed orbital models. The problem's domain is restricted to LEO orbits, and a time horizon of a few revolutions. Here, most efficient is defined as least propellant consumption. No assumption of circular or coplanar orbits, or any other simplifying assumption about initial and final states is allowed in the modelling and algorithms of this work. The work is restricted to finding satisfactory local optima, instead of trying to characterize the global optimum or its lower bounds.

A central topic to be explored is the question of how many impulses are needed for a particular maneuver. This is the most classic question in the field, but still a topic of research~\cite{how_many_impulses}. This matter shall be explored with primer vector theory~\cite{Conway_2010}, which is often presented for a two-body Keplerian orbital model. This theory shall be extended to encompass relevant orbital disturbances in LEO maneuvering scenarios.

The proposed scenarios in this work are a circular-to-circular rendez-vous problem and a non-coplanar rendez-vous in multiple revolutions. The goal is to validate the results under the Keplerian model with the results for the known Hohmann transfer~\cite{curtis2015orbital} and previous results in the primer vector field, as in~\citeonline{interactive_primer_vector}. Then, the same scenarios are to be solved under perturbed orbital models, including oblateness effects and drag, extending the domain of the primer vector method to new models.

\section{Hypotheses}

The impulsive maneuver optimization problem is expected to be reducible to a simple parameter optimization problem where impulse parameters are directly optimized in modern non-linear optimizers with a multiple shooting scheme, which is standard in optimal control and in the field of impulsive maneuvers~\cite{impulsive_europa}. The need for numerical solutions is well established and taken as granted. However, it is expected that many local optima are to be found, due to the non-convex nature of direct optimal control problems and of the domain of the state vector, and the periodic motion of satellites, which makes multi-revolution transfers particularly challenging.

The time of transfer is frequently treated as a fixed input parameter, both in the theory~\cite{Conway_2010} and in research~\cite{fixed_time_primer_vector}, since it can prevent optimizers from generating infinite-time maneuvers~\cite{impulsive_europa}. The longer the transfer time, the more revolutions around the central body are expected, and the more (small) impulses are expected. The more revolutions are allowed, the more local optima of bad performance are expected, requiring more computational work to arrive at satisfactory solution.

Primer vector theory is expected to be useful for determining how many impulses are necessary, up to the question of whether a solution that satisfies the necessary conditions is actually a satisfactory solution. The problem is known to have many local optima~\cite{interactive_primer_vector}, but it is expected that primer vector theory will always improve, or maintain performance, of a given suboptimal solution by adding or removing impulses~\cite{Conway_2010}. That is, primer vector theory should produce a decreasing sequence of costs.

The primer vector literature under the two body model is extensive, with some works applying it to the restricted three body problem~\cite{impulsive_europa}. However, no work was found applying this theory to orbital perturbations in LEO, namely Earth's oblateness effects and atmospheric drag. Therefore, it is expected that some adaptation of the theory is in order to apply the primer vector method to perturbed models. 

% Finally, due to the existence of many suboptimal solutions, and the expected usage of local optimizers, it is expected that solutions will be locally optimal with no guarantee of global optimality. Ensuring global optimality is a much harder problem and, except in particular cases where physical reasoning may give insight, or the mathematical structure of the problem allows for a systematic lower bound estimation.



\section{Objectives}

The objective of this work is to extend to perturbed orbital models the primer vector method for finding the optimal sequence of impulses that transfer a satellite from an arbitrary starting orbit to an arbitrary final orbit, in a given amount of time, and apply it for some practical scenarios. Secondary objectives include:
\begin{itemize}
    % \item Apply primer vector theory to the solutions found. This is a central tool in the field, and provides analytical necessary conditions for verifying optimality;
    % \item Study how much time is needed to execute the proposed transfers;
    % \item Compare some of the numerical results with known analytical results, namely the Hohmann transfer;
    % \item Discuss some instances of application of this method to common aerospace scenarios;
    % \item \textbf{Optional}: expand the work to continuous thrust propulsive models;
    % \item \textbf{Optional}: include orbital perturbations, most notably oblateness effects.
    \item Compare maneuvers with known solutions under the standard two body model with their perturbed-model counterparts;
    \item Describe different methods for calculating the primer vector, and the domain of validity of each one according to the nature of the orbital model;
    \item Present a multiple-shooting algorithm for optimizing maneuvers with a fixed number of impulses for any orbital model;
    \item Discuss the limitations of the presented methods in terms of problem size, numerical accuracy, and solver characteristics.
\end{itemize}

% \section{Justification}

% The strict performance requirements in the space domain make it paramount to use orbital resources sparingly. Optimization of orbital maneuvers increases the envelope of possible missions, be it in terms of mission lifespan, which increases profitability, or in terms of mission design, allowing for bolder, high-profile missions. 

% In the context of Brazil's space industry, ITASAT-2 is a formation-flying mission which is subject to orbital disturbances, such as atmospheric drag and oblateness effects~\cite{itasat2}. Thus the need for efficient orbital maneuvers arises.


\section{Work structure}

This work is organized in chapters as follows:
\begin{enumerate}
    \item \textbf{Introduction}, where preliminary contextualization is given;
    \item \textbf{Theory} and fundamentals, where the mathematical description of the problem is given;
    \item \textbf{Bibliographic review}, where some previous results in the field are discussed;
    \item \textbf{Methodology}, where implementation details are given;
    \item \textbf{Results}, where numerical scenarios exemplify the proposed method, and comparisons with known results are made;
    \item \textbf{Conclusion}, where the merits and issues of this work are summarized, and future investigations are proposed.
\end{enumerate}