
The primer vector meta-algorithm was applied to some scenarios that represent generic maneuvers that may be of interest in LEO\@. These scenarios will be presented, together with an explanation of why they were chosen, and then solutions for each of them, under different orbital models, will be presented.
%TODO discuss orbit unraveling, heavy drag models
%TODO change Y+ -> Y-
\section{Maneuver scenarios}

The scenarios of orbital maneuvering used in this work are introduced in this section before their solutions under different orbital models. Two scenarios were chosen, one where a coplanar transfer between two circular orbits is done in half a revolution, and a noncoplanar rendez-vous case spanning two full revolutions. Both cases are presented in detail ahead.

\subsection{Circle to Circle rendez-vous}

A maneuver scenario corresponding to a Hohmann transfer (under a Keplerian model) was chosen. Firstly, because in the Keplerian model, there is an analytical solution for the transfer, given in Section~\ref{sssec:hohmann}. And secondly, this is a simple scenario that highlights the difference between the different orbital models used in the work. The orbital parameters are given in Table~\ref{tab:c2c_orb_elems}. The transfer time was calculated analytically to correspond to the Hohmann transfer time, to verify that the Hohmann transfer is recovered by the primer vector meta-algorithm in the Keplerian model. This scenario corresponds to a large orbit raising problem with 2000 km of semi-major axis difference.

% C2C SETUP
\begin{table}[htbp]
    \centering
    \begin{tabular}{ccc} \toprule
        Element & Initial & Final \\ \midrule
        \(a\)      & \(7000.0\) km         & \(9000.0\) km   \\
        \(e\)      & \(0.0\)            & \(0.0\)        \\
        \(i\)      & \(51.0^\circ\)      & \(51.0^\circ\) \\
        \(\Omega\) & \(0.0^\circ\)   & \(0.0^\circ\)  \\
        \(\omega\) & \(0.0^\circ\)  & \(0.0^\circ\)  \\
        \(\theta\) & \(0.0^\circ\)  & \(180.0^\circ\)  \\ 
        Transfer time & \multicolumn{2}{c}{\(3560.541\)} \\\bottomrule
    \end{tabular}
    \caption{Orbital elements used for the Circle to circle transfer case analysis}
    \label{tab:c2c_orb_elems}
\end{table}
% C2C SETUP


% \begin{figure}[htbp]
%     \centering
%     \includegraphics[width=0.7\textwidth]{../results/two_body/hohmann/scenario.png}
%     \caption{Circle to Circle transfer scenario.}
%     \label{fig:hohmann_scenario}
% \end{figure}
% \FloatBarrier
\subsection{Noncoplanar rendez-vous}

TODO compare with table from paper

A second, more challenging scenario is taken from \citeonline{interactive_primer_vector}, where a noncoplanar rendez-vous scenario is proposed and solved under the Keplerian model. Again, the goal is validating the code against a published result under the Keplerian model, and exploring how the optimal trajectory may be different under other more realistic models. The orbital parameters are given in Table~\ref{tab:noncop_rdv_orb_elems}. The transfer time is equal to two orbital periods of the target orbit. This is a small maneuver in a medium-term scenario (2 revolutions) which will test the numerical stability of the algorithms used.

% NCOP SCENARIO
\begin{table}[htbp]
    \centering
    \begin{tabular}{ccc} \toprule
        Element & Initial & Final \\ \midrule
        \(a\)      & \(6748.1\) km         & \(6778.1\) km   \\
        \(e\)      & \(0.0\)            & \(0.0\)        \\
        \(i\)      & \(42.1^\circ\)      & \(42.0^\circ\) \\
        \(\Omega\) & \(120.2^\circ\)   & \(120.0^\circ\)  \\
        \(\omega\) & \(0.0^\circ\)  & \(0.0^\circ\)  \\
        \(\theta\) & \(175.0^\circ\)  & \(180.0^\circ\)  \\ 
        Transfer time & \multicolumn{2}{c}{\(11107.158\)} \\\bottomrule
    \end{tabular}
    \caption{Orbital elements used for the noncoplanar rendez-vous transfer case analysis}
    \label{tab:noncop_rdv_orb_elems}
\end{table}
% NCOP SCENARIO


% \begin{figure}[htbp]
%     \centering
%     \begin{subfigure}{0.49\linewidth}
%         \includegraphics[width=0.9\linewidth]{../results/two_body/ipv_noncop/scenario_3d.png}
%         \caption{3D view.}
%     \end{subfigure}
%     \begin{subfigure}{0.49\linewidth}
%         \includegraphics[width=0.9\linewidth]{../results/two_body/ipv_noncop/scenario_x+.png}
%         \caption{View from x+ axis.}
%     \end{subfigure}
%     \begin{subfigure}{0.49\linewidth}
%         \includegraphics[width=0.9\linewidth]{../results/two_body/ipv_noncop/scenario_y-.png}
%         \caption{View from y- axis.}
%     \end{subfigure}
%     \begin{subfigure}{0.49\linewidth}
%         \includegraphics[width=0.9\linewidth]{../results/two_body/ipv_noncop/scenario_z+.png}
%         \caption{View from z+ axis.}
%     \end{subfigure}
%     \caption{Noncoplanar rendez-vous scenario 3D view and projections}
%     \label{fig:noncop_rdv_scenario}
% \end{figure}

\FloatBarrier
\section{Keplerian Model}

The results for all scenarios under the Keplerian model are presented in the following. All primer vector histories in this section were calculated with the Glandorf, STM and ODE methods, to numerically show they are equivalent. In all primer vector plots, vertical dashed lines indicate impulse times.

\subsection{Circle to Circle}

The circle to circle transfer scenario was first optimized for an \texttt{ICI} maneuver sequence. The views of the optimized trajectory can be seen in Figure~\ref{fig:tb_c2c_ICI_figs}, the primer vector trajectory can be seen in Figure~\ref{fig:tb_c2c_ICI_pv}, and the numerical results, in Table~\ref{tab:tb_ctcr_ICI_tab}. In the spatial views, the blue orbit is the initial orbit, the yellow orbit is the final orbit, with the blue and yellow dots marking the initial and final states. Red lines show impulse vectors and their locations, and the green dots are the positions of the discretized states returned by the solver. In the primer vector plots, the vertical dashed lines show impulse instants, and the dashed blue, dash-dotted yellow and dash-dot-dotted green lines show the primer vector trajectories computed from the Glandorf, STM and ODE methods respectively.

From the primer vector plot, it can be seen that all three primer vector calculation methods are equivalent, as predicted, under the Keplerian model. Impulses at the start and end times are capable of satisfying the necessary optimality conditions, since the primer vector norm is always less than unity and its derivative is continuous. In addition, the derivative of the primer vector norm is zero at the start and end positions, meaning that if this maneuver is to be considered a \texttt{CICIC} maneuver with zero duration coasts at the start and end, it is also a local optimum. Therefore, the procedure is stopped. Under the Keplerian model, this is a Hohmann transfer, with analytical cost of \(\Delta v_H = 887.56199 m/s\), computed from Equation~\eqref{eq:hohmann_deltav}, identical to the optimized cost. The problem was solved with length rescaling and no time rescaling. The scaling factors, together with the solver tolerance \(\varepsilon\), determine the tolerance on impulse magnitudes. For impulses on the order of \(10^2-10^3\) m/s, this configuration was found to yield the best results.

% TB C2C ICI
\begin{table}[htpb]
    \centering
    \begin{tabular}{cccc} \toprule
    \multicolumn{2}{c}{\textbf{Maneuver type}} & \multicolumn{2}{c}{\texttt{ICI}} \\ \midrule
    \(L\) (m) & \(T\) (s) & \(\varepsilon\) & \(\lVert \Delta \pos_{f} \rVert\) (m)    \\ \midrule
    8.0e6          & 1.0          & 1.00e-05                & 2.11019e-08                        \\ \midrule
    \(\max \lVert p \rVert\) & 1.0     & \textbf{Diagnostic}   & Local optimum        \\ \midrule
    \textbf{Impulse} & \(t\) (s) & \multicolumn{2}{c}{\(\Delta v\) (m/s)} \\ \midrule
    1                 & 0.0          & \multicolumn{2}{c}{457.74489}                  \\
    2                 & 3560.54079          & \multicolumn{2}{c}{429.8171}                  \\\midrule
    \textbf{Total}   & 3560.54079          & \multicolumn{2}{c}{887.56199}\\ \bottomrule   
    \end{tabular}
    \caption{Summary of optimization for \texttt{ICI} Keplerian Circle to circle RV.}
    \label{tab:tb_ctcr_ICI_tab}
\end{table}
% TB C2C ICI

\begin{figure}[htbp]
    \centering
    \begin{subfigure}{0.49\linewidth}
        \includegraphics[width=0.9\linewidth]{../results/two_body/hohmann/ICI_3d.png}
        \caption{3D view.}
    \end{subfigure}
    \begin{subfigure}{0.49\linewidth}
        \includegraphics[width=0.9\linewidth]{../results/two_body/hohmann/ICI_x+.png}
        \caption{View from x+ axis.}
    \end{subfigure}
    \begin{subfigure}{0.49\linewidth}
        \includegraphics[width=0.9\linewidth]{../results/two_body/hohmann/ICI_y-.png}
        \caption{View from y- axis.}
    \end{subfigure}
    \begin{subfigure}{0.49\linewidth}
        \includegraphics[width=0.9\linewidth]{../results/two_body/hohmann/ICI_z+.png}
        \caption{View from z+ axis.}
    \end{subfigure}
    \caption{Circle to circle \texttt{ICI} maneuver 3D view and projections}
    \label{fig:tb_c2c_ICI_figs}
\end{figure}

\begin{figure}[htbp]
    \centering
    \includegraphics[width=0.9\linewidth]{../results/two_body/hohmann/ICI_primer_vector.png}
    \caption{Primer vector trajectory for Keplerian circle to circle \texttt{ICI} rendez-vous.}
    \label{fig:tb_c2c_ICI_pv}
\end{figure}

\subsection{Noncoplanar rendez-vous}

The more complex noncoplanar rendez-vous case was optimized with an \texttt{ICI} maneuver at first. The maneuver's summary, spatial views, and primer vector trajectory can be seen in Table~\ref{tab:Keplerian_nr_ICI_tab}, Figure~\ref{fig:tb_ncop_ICI_figs} and Figure~\ref{fig:tb_ncop_ICI_pv}. The primer vector history shows that this is an extremum, since its norm is less than one everywhere, but physical intuition suggests that there are less costly maneuvers that can solve the problem, since this maneuver requires a large deviation from the natural trajectory of the satellite. Even though the initial and final states are quite close in space, they are separated by differences in all orbital elements. The initial and final orbits are on different planes and have different semi-major axes. All of this means that no \texttt{ICI} maneuver with two revolutions has been found, which makes unlikely (but does not rule out) the possibility that such a maneuver exists. In order to find better optima, the problem must be relaxed: the introduction of initial and final coasts increases the degrees of freedom of the control, possibly allowing for a better cost.

% TB NCOP ICI
\begin{table}[htpb]
    \centering
    \begin{tabular}{cccc} \toprule
    \multicolumn{2}{c}{\textbf{Maneuver type}} & \multicolumn{2}{c}{\texttt{ICI}} \\ \midrule
    \(L\) (m) & \(T\) (s) & \(\varepsilon\) & \(\lVert \Delta \pos_{f} \rVert\) (m)    \\ \midrule
    6.7631e6          & 1.0          & 1.00e-06                & 4.26292e-06                        \\ \midrule
    \(\max \lVert p \rVert\) & 1.0     & \textbf{Diagnostic}   & Local optimum        \\ \midrule
    \textbf{Impulse} & \(t\) (s) & \multicolumn{2}{c}{\(\Delta v\) (m/s)} \\ \midrule
    1                 & 0.0          & \multicolumn{2}{c}{11740.94035}                  \\
    2                 & 11107.1576          & \multicolumn{2}{c}{11708.69678}                  \\\midrule
    \textbf{Total}   & 11107.1576          & \multicolumn{2}{c}{23449.63713}\\ \bottomrule   
    \end{tabular}
    \caption{Summary of optimization for \texttt{ICI} Keplerian noncoplanar rendez-vous.}
    \label{tab:Keplerian_nr_ICI_tab}
\end{table}
% TB NCOP ICI

\begin{figure}[htbp]
    \centering
    \begin{subfigure}{0.49\linewidth}
        \includegraphics[width=0.9\linewidth]{../results/two_body/ipv_noncop/ICI_3d.png}
        \caption{3D view.}
    \end{subfigure}
    \begin{subfigure}{0.49\linewidth}
        \includegraphics[width=0.9\linewidth]{../results/two_body/ipv_noncop/ICI_x+.png}
        \caption{View from x+ axis.}
    \end{subfigure}
    \begin{subfigure}{0.49\linewidth}
        \includegraphics[width=0.9\linewidth]{../results/two_body/ipv_noncop/ICI_y-.png}
        \caption{View from y- axis.}
    \end{subfigure}
    \begin{subfigure}{0.49\linewidth}
        \includegraphics[width=0.9\linewidth]{../results/two_body/ipv_noncop/ICI_z+.png}
        \caption{View from z+ axis.}
    \end{subfigure}
    \caption{Noncoplanar rendez-vous \texttt{ICI} maneuver 3D view and projections}
    \label{fig:tb_ncop_ICI_figs}
\end{figure}

\begin{figure}[htbp]
    \centering
    \includegraphics[width=0.9\linewidth]{../results/two_body/ipv_noncop/ICI_primer_vector.png}
    \caption{Primer vector trajectory for Keplerian noncoplanar \texttt{ICI} rendez-vous.}
    \label{fig:tb_ncop_ICI_pv}
\end{figure}
\FloatBarrier
Therefore, a \texttt{CICIC} maneuver case was optimized next. The maneuver's summary, spatial views, and primer vector trajectory can be seen in Table~\ref{tab:Keplerian_nr_CICIC_tab}, Figure~\ref{fig:tb_ncop_CICIC_figs} and Figure~\ref{fig:tb_ncop_CICIC_pv}. By allowing impulses to happen at any time, they can more effectively change the plane of the orbit and the cost is much lower, and the trajectory is very close to the natural orbit of the satellite. The spatial views show, with the arrows in red, that the impulses do not happen in the plane of the orbit, showing that they change the plane of the trajectory. However, the primer vector history violates the norm condition and shows that the addition of an extra impulse can lower the cost even further.

% TB NCOP 2 CICIC
\begin{table}[htpb]
    \centering
    \begin{tabular}{cccc} \toprule
    \multicolumn{2}{c}{\textbf{Maneuver type}} & \multicolumn{2}{c}{\texttt{CICIC}} \\ \midrule
    \(L\) (m) & \(T\) (s) & \(\varepsilon\) & \(\lVert \Delta \pos_{f} \rVert\) (m)    \\ \midrule
    6.7631e6          & 11107.158          & 1.00e-06                & 1.66471e-07                        \\ \midrule
    \(\max \lVert p \rVert\) & 3.327     & \textbf{Diagnostic}   & Add impulse        \\ \midrule
    \textbf{Impulse} & \(t\) (s) & \multicolumn{2}{c}{\(\Delta v\) (m/s)} \\ \midrule
    1                 & 6644.30733          & \multicolumn{2}{c}{37.29252}                  \\
    2                 & 10689.86179          & \multicolumn{2}{c}{16.20984}                  \\\midrule
    \textbf{Total}   & 11107.1576          & \multicolumn{2}{c}{53.50237}\\ \bottomrule   
    \end{tabular}
    \caption{Summary of optimization for \texttt{CICIC} Keplerian noncoplanar rendez-vous.}
    \label{tab:Keplerian_nr_CICIC_tab}
\end{table}
% TB NCOP 2 CICIC

\begin{figure}[htbp]
    \centering
    \begin{subfigure}{0.49\linewidth}
        \includegraphics[width=0.9\linewidth]{../results/two_body/ipv_noncop/CICIC_3d.png}
        \caption{3D view.}
    \end{subfigure}
    \begin{subfigure}{0.49\linewidth}
        \includegraphics[width=0.9\linewidth]{../results/two_body/ipv_noncop/CICIC_x+.png}
        \caption{View from x+ axis.}
    \end{subfigure}
    \begin{subfigure}{0.49\linewidth}
        \includegraphics[width=0.9\linewidth]{../results/two_body/ipv_noncop/CICIC_y-.png}
        \caption{View from y- axis.}
    \end{subfigure}
    \begin{subfigure}{0.49\linewidth}
        \includegraphics[width=0.9\linewidth]{../results/two_body/ipv_noncop/CICIC_z+.png}
        \caption{View from z+ axis.}
    \end{subfigure}
    \caption{Noncoplanar rendez-vous \texttt{CICIC} maneuver 3D view and projections}
    \label{fig:tb_ncop_CICIC_figs}
\end{figure}

\begin{figure}[htbp]
    \centering
    \includegraphics[width=0.9\linewidth]{../results/two_body/ipv_noncop/CICIC_primer_vector.png}
    \caption{Primer vector trajectory for Keplerian noncoplanar \texttt{CICIC} rendez-vous.}
    \label{fig:tb_ncop_CICIC_pv}
\end{figure}
\FloatBarrier
A \texttt{CICICIC} maneuver was searched based on the previous results. The spatial view is omitted since the trajectory closely resembles the natural orbit of the satellite. The maneuver summary and primer vector history are shown in Table~\ref{tab:Keplerian_nr_CICICIC_tab} and Figure~\ref{fig:tb_ncop_CICICIC_pv}. The addition of yet another impulse did reduce the \(\Delta v\) cost, but not as dramatically as the addition of initial and final coasts. Again, the all necessary conditions but the norm condition are verified. The maximum primer vector norm exceeds unity, which suggests another impulse addition.

% TB NCOP 3 CICICIC
\begin{table}[htpb]
    \centering
    \begin{tabular}{cccc} \toprule
    \multicolumn{2}{c}{\textbf{Maneuver type}} & \multicolumn{2}{c}{\texttt{CICICIC}} \\ \midrule
    \(L\) (m) & \(T\) (s) & \(\varepsilon\) & \(\lVert \Delta \pos_{f} \rVert\) (m)    \\ \midrule
    6.7631e6          & 11107.158          & 1.00e-05                & 1.08852e-07                        \\ \midrule
    \(\max \lVert p \rVert\) & 1.9638     & \textbf{Diagnostic}   & Add impulse        \\ \midrule
    \textbf{Impulse} & \(t\) (s) & \multicolumn{2}{c}{\(\Delta v\) (m/s)} \\ \midrule
    1                 & 3370.1071          & \multicolumn{2}{c}{11.76294}                  \\
    2                 & 6774.61652          & \multicolumn{2}{c}{10.56494}                  \\
    3                 & 9176.42663          & \multicolumn{2}{c}{20.74554}                  \\\midrule
    \textbf{Total}   & 11107.1576          & \multicolumn{2}{c}{43.07342}\\ \bottomrule   
    \end{tabular}
    \caption{Summary of optimization for \texttt{CICICIC} Keplerian noncoplanar rendez-vous.}
    \label{tab:Keplerian_nr_CICICIC_tab}
\end{table}
% TB NCOP 3 CICICIC

\begin{figure}[htbp]
    \centering
    \includegraphics[width=0.9\linewidth]{../results/two_body/ipv_noncop/CICICIC_primer_vector.png}
    \caption{Primer vector trajectory for Keplerian noncoplanar \texttt{CICICIC} rendez-vous.}
    \label{fig:tb_ncop_CICICIC_pv}
\end{figure}

Finally, the addition of a fourth impulse attains the primer vector necessary conditions, as attested in its summary in Table~\ref{tab:Keplerian_nr_CICICICIC_tab} and in its trajectory in Figure~\ref{fig:tb_ncop_CICICICIC_pv}. This maneuver is similar to the previous one, in that its impulses 2 and 3 of the 4-impulse maneuver happen at approximately the same times and have the same magnutides as impulses 2 and 3 of the 3-impulse maneuver. This likely means that those instants are close to relative ascending nodes, and are very physically relevant for the maneuver. A small excess of \(0.4\%\) was found in the primer vector norm, meaning that a fifth impulse could reduce the cost further. However, given the magnitude of this excess, and the fact that the final velocity cost is well within feasible for many spatial applications, the algorithm was stopped at four impulses. 

% TB NCOP 4 CICICICIC
\begin{table}[htpb]
    \centering
    \begin{tabular}{cccc} \toprule
    \multicolumn{2}{c}{\textbf{Maneuver type}} & \multicolumn{2}{c}{\texttt{CICICICIC}} \\ \midrule
    \(L\) (m) & \(T\) (s) & \(\varepsilon\) & \(\lVert \Delta \pos_{f} \rVert\) (m)    \\ \midrule
    6.7631e6          & 11107.158          & 1.00e-05                & 4.98323e-02                        \\ \midrule
    \(\max \lVert p \rVert\) & 1.0046     & \textbf{Diagnostic}   & Add impulse        \\ \midrule
    \textbf{Impulse} & \(t\) (s) & \multicolumn{2}{c}{\(\Delta v\) (m/s)} \\ \midrule
    1                 & 0.00394          & \multicolumn{2}{c}{3.58823}                  \\
    2                 & 6724.60052          & \multicolumn{2}{c}{9.34687}                  \\
    3                 & 9217.50949          & \multicolumn{2}{c}{15.01486}                  \\
    4                 & 11107.15747          & \multicolumn{2}{c}{8.19599}                  \\\midrule
    \textbf{Total}   & 11107.1576          & \multicolumn{2}{c}{36.14596}\\ \bottomrule   
    \end{tabular}
    \caption{Summary of optimization for \texttt{CICICICIC} Keplerian noncoplanar rendez-vous.}
    \label{tab:Keplerian_nr_CICICICIC_tab}
\end{table}
% TB NCOP 4 CICICICIC

\begin{figure}[htbp]
    \centering
    \includegraphics[width=0.9\linewidth]{../results/two_body/ipv_noncop/CICICICIC_primer_vector.png}
    \caption{Primer vector trajectory for Keplerian noncoplanar \texttt{CICICICIC} rendez-vous.}
    \label{fig:tb_ncop_CICICICIC_pv}
\end{figure}

Finally, all cases are summarized in Table~\ref{tab:Keplerian_nr_summary}, where the monotonic cost reduction obtained by following the primer vector algorithm is evident. Figure~\ref{fig:ipv_table} shows the results presented in~\citeonline{interactive_primer_vector}, the source article for this scenario. The results computed here are quite similar to those found there, but not exactly the same. Different local optima can be found by different algorithms, and that article did not disclose solver tolerances or configuration, which prevents an exact comparison. However, the similarity in values validates the algorithm, and allows the exploratory work of extending it to perturbed models. 

% TB NCOP SUM
\begin{table}[htbp]
    \centering
    \begin{tabular}{ccc} \toprule
        Maneuver type & Number of impulses & Cost (m/s) \\ \midrule
        \texttt{ICI} & 2 & 23449.63713 \\
        \texttt{CICIC} & 2 & 53.50237 \\
        \texttt{CICICIC} & 3 & 43.07342 \\
        \texttt{CICICICIC} & 4 & 36.14596 \\\bottomrule
    \end{tabular}
    \caption{Summary of Keplerian noncoplanar rendez-vous maneuvers and costs.}
    \label{tab:Keplerian_nr_summary}
\end{table}
% TB NCOP SUM

\begin{figure}[htbp]
    \centering
    \includegraphics[width=\linewidth]{img/ipv_table.png}
    \caption{Results table from the noncoplanar rendez-vous case presented by~\citeonline{interactive_primer_vector}.}
    \label{fig:ipv_table}
\end{figure}

\newpage
\FloatBarrier
\section{J2 Model}

Optimized maneuvers are presented for the aforementioned scenarios under the J2 model introduced previously. Under this model, only the STM and ODE methods correctly calculate primer vector trajectories, and both will be presented for all cases.

\subsection{Circle to Circle}


The simple circle to circle rendez-vous case was first optimized with a \texttt{ICI} maneuver, with results that can be seen in Table~\ref{tab:J2_ctcr_ICI_tab}, Figure~\ref{fig:j2_c2c_ICI_figs} and Figure~\ref{fig:j2_c2c_ICI_pv}. By comparing Figures~\ref{fig:j2_c2c_ICI_figs} and~\ref{fig:tb_c2c_ICI_figs}, it can be seen that the trajectories are very different. The cost of \(9528.2\) m/s for the maneuver under the J2 model, compared to \(887.6\) m/s in the Keplerian model, highlights the difference in trajectories and thus, the necessity of modelling orbital perturbations, even in short time frames.

To analyze the polar trajectory that was found, first it is worth mentioning that orbital trajectories under the J2 model are, in general, not planar, which means that the Keplerian Hohmann transfer is impossible with this model. The only planar orbits are perfectly equatorial and perfectly polar orbits and, due to the fact that the inclinations of the orbits' are greater than \(45^\circ\), the polar orbit is the cheaper solution. The primer vector trajectories in this section show that the STM and ODE methods are equivalent for the conservatively perturbed J2 model, and early testing showed that the differ from the Glandorf method, not applicable to this model, as expected. However, the endpoint primer vector norm derivatives are positive and negative, respectively, suggesting the addition of initial and final coasts. This \texttt{ICI} solution is not even a local optimum in the space of \texttt{CICIC} solutions. Therefore, adding initial and final coasts should reduce the very large cost of this solution.

% J2 C2C ICI
\begin{table}[htpb]
    \centering
    \begin{tabular}{cccc} \toprule
    \multicolumn{2}{c}{\textbf{Maneuver type}} & \multicolumn{2}{c}{\texttt{ICI}} \\ \midrule
    \(L\) (m) & \(T\) (s) & \(\varepsilon\) & \(\lVert \Delta \pos_{f} \rVert\) (m)    \\ \midrule
    8.0e6          & 1.0          & 1.00e-05                & 4.97128e-01                        \\ \midrule
    \(\max \lVert p \rVert\) & 563.46     & \textbf{Diagnostic}   & Initial + Final coast        \\ \midrule
    \textbf{Impulse} & \(t\) (s) & \multicolumn{2}{c}{\(\Delta v\) (m/s)} \\ \midrule
    1                 & 0.0          & \multicolumn{2}{c}{5209.47789}                  \\
    2                 & 3560.54079          & \multicolumn{2}{c}{4318.71793}                  \\\midrule
    \textbf{Total}   & 3560.54079          & \multicolumn{2}{c}{9528.19582}\\ \bottomrule   
    \end{tabular}
    \caption{Summary of optimization for \texttt{ICI} J2 Circle to circle RV.}
    \label{tab:J2_ctcr_ICI_tab}
\end{table}
% J2 C2C ICI

\begin{figure}[htbp]
    \centering
    \begin{subfigure}{0.49\linewidth}
        \includegraphics[width=0.9\linewidth]{../results/j2/hohmann/ICI_3d.png}
        \caption{3D view.}
    \end{subfigure}
    \begin{subfigure}{0.49\linewidth}
        \includegraphics[width=0.9\linewidth]{../results/j2/hohmann/ICI_x+.png}
        \caption{View from x+ axis.}
    \end{subfigure}
    \begin{subfigure}{0.49\linewidth}
        \includegraphics[width=0.9\linewidth]{../results/j2/hohmann/ICI_y-.png}
        \caption{View from y- axis.}
    \end{subfigure}
    \begin{subfigure}{0.49\linewidth}
        \includegraphics[width=0.9\linewidth]{../results/j2/hohmann/ICI_z+.png}
        \caption{View from z+ axis.}
    \end{subfigure}
    \caption{Circle to circle \texttt{ICI} maneuver 3D view and projections}
    \label{fig:j2_c2c_ICI_figs}
\end{figure}

\begin{figure}[htbp]
    \centering
    \includegraphics[width=0.9\linewidth]{../results/j2/hohmann/ICI_primer_vector.png}
    \caption{Primer vector trajectory for J2 circle to circle \texttt{ICI} rendez-vous.}
    \label{fig:j2_c2c_ICI_pv}
\end{figure}
\FloatBarrier
The solution of the \texttt{CICIC} maneuver is shown in Table~\ref{tab:J2_ctcr_CICIC_tab}, Figure~\ref{fig:j2_c2c_CICIC_figs} and Figure~\ref{fig:j2_c2c_CICIC_pv}. In the spatial views in Figure~\ref{fig:j2_c2c_CICIC_figs}, the trajectory now closely resembles a Hohmann transfer, with the big difference that the impulses are not aligned with the velocities and that there are short initial and final coasts. The cost is now comparable with the Hohmann transfer cost, albeit slightly higher. Again, these differences can be explained mainly by the non-planarity of J2 coasting arcs. The primer vector trajectory now has zero endpoint derivatives but its norm exceeds unity, showing that this trajectory is not a local optimum in the \texttt{CICICIC} maneuver space, so adding another impulse should decrease the cost.

% J2 C2C 2 CICIC
\begin{table}[htpb]
    \centering
    \begin{tabular}{cccc} \toprule
    \multicolumn{2}{c}{\textbf{Maneuver type}} & \multicolumn{2}{c}{\texttt{CICIC}} \\ \midrule
    \(L\) (m) & \(T\) (s) & \(\varepsilon\) & \(\lVert \Delta \pos_{f} \rVert\) (m)    \\ \midrule
    8.0e6          & 1.0          & 1.00e-05                & 4.16406e-05                        \\ \midrule
    \(\max \lVert p \rVert\) & 2.0864     & \textbf{Diagnostic}   & Add impulse        \\ \midrule
    \textbf{Impulse} & \(t\) (s) & \multicolumn{2}{c}{\(\Delta v\) (m/s)} \\ \midrule
    1                 & 72.53153          & \multicolumn{2}{c}{473.46697}                  \\
    2                 & 3448.63865          & \multicolumn{2}{c}{438.46605}                  \\\midrule
    \textbf{Total}   & 3560.54079          & \multicolumn{2}{c}{911.93302}\\ \bottomrule   
    \end{tabular}
    \caption{Summary of optimization for \texttt{CICIC} J2 Circle to circle RV.}
    \label{tab:J2_ctcr_CICIC_tab}
\end{table}
% J2 C2C 2 CICIC

\begin{figure}[htbp]
    \centering
    \begin{subfigure}{0.49\linewidth}
        \includegraphics[width=0.9\linewidth]{../results/j2/hohmann/CICIC_3d.png}
        \caption{3D view.}
    \end{subfigure}
    \begin{subfigure}{0.49\linewidth}
        \includegraphics[width=0.9\linewidth]{../results/j2/hohmann/CICIC_x+.png}
        \caption{View from x+ axis.}
    \end{subfigure}
    \begin{subfigure}{0.49\linewidth}
        \includegraphics[width=0.9\linewidth]{../results/j2/hohmann/CICIC_y-.png}
        \caption{View from y- axis.}
    \end{subfigure}
    \begin{subfigure}{0.49\linewidth}
        \includegraphics[width=0.9\linewidth]{../results/j2/hohmann/CICIC_z+.png}
        \caption{View from z+ axis.}
    \end{subfigure}
    \caption{Circle to circle \texttt{CICIC} maneuver 3D view and projections}
    \label{fig:j2_c2c_CICIC_figs}
\end{figure}

\begin{figure}[htbp]
    \centering
    \includegraphics[width=0.9\linewidth]{../results/j2/hohmann/CICIC_primer_vector.png}
    \caption{Primer vector trajectory for J2 circle to circle \texttt{CICIC} rendez-vous.}
    \label{fig:j2_c2c_CICIC_pv}
\end{figure}
\FloatBarrier
Now, the \texttt{CICICIC} case is solved, with a summary shown in Table~\ref{tab:J2_ctcr_CICICIC_tab}, spatial views shown in Figure~\ref{fig:j2_c2c_CICICIC_figs}, and primer vector trajectory shown in Figure~\ref{fig:j2_c2c_CICICIC_pv}. The trajectory is now almost identical to that of a Hohmann transfer, with initial and final impulses almost parallel to the velocity.A very small intermediate impulse is made about halfway through the maneuver, which is the most remarkable difference with respect to the equivalent maneuver in the Keplerian model. The necessary conditions on the primer vector trajectory for optimality are satisfied with 3 impulses, and the final cost of the J2 model is about \(6 m/s\) higher than under the Keplerian model.

% J2 C2C 3 CICICIC
\begin{table}[htpb]
    \centering
    \begin{tabular}{cccc} \toprule
    \multicolumn{2}{c}{\textbf{Maneuver type}} & \multicolumn{2}{c}{\texttt{CICICIC}} \\ \midrule
    \(L\) (m) & \(T\) (s) & \(\varepsilon\) & \(\lVert \Delta \pos_{f} \rVert\) (m)    \\ \midrule
    8.0e6          & 1.0          & 1.00e-05                & 2.20114e-02                        \\ \midrule
    \(\max \lVert p \rVert\) & 1.0     & \textbf{Diagnostic}   & Local optimum        \\ \midrule
    \textbf{Impulse} & \(t\) (s) & \multicolumn{2}{c}{\(\Delta v\) (m/s)} \\ \midrule
    1                 & 0.38763          & \multicolumn{2}{c}{451.26267}                  \\
    2                 & 1697.05494          & \multicolumn{2}{c}{25.16996}                  \\
    3                 & 3559.91404          & \multicolumn{2}{c}{416.62074}                  \\\midrule
    \textbf{Total}   & 3560.54079          & \multicolumn{2}{c}{893.05336}\\ \bottomrule   
    \end{tabular}
    \caption{Summary of optimization for \texttt{CICICIC} J2 Circle to circle RV.}
    \label{tab:J2_ctcr_CICICIC_tab}
\end{table}
% J2 C2C 3 CICICIC

\begin{figure}[htbp]
    \centering
    \begin{subfigure}{0.49\linewidth}
        \includegraphics[width=0.9\linewidth]{../results/j2/hohmann/CICICIC_3d.png}
        \caption{3D view.}
    \end{subfigure}
    \begin{subfigure}{0.49\linewidth}
        \includegraphics[width=0.9\linewidth]{../results/j2/hohmann/CICICIC_x+.png}
        \caption{View from x+ axis.}
    \end{subfigure}
    \begin{subfigure}{0.49\linewidth}
        \includegraphics[width=0.9\linewidth]{../results/j2/hohmann/CICICIC_y-.png}
        \caption{View from y- axis.}
    \end{subfigure}
    \begin{subfigure}{0.49\linewidth}
        \includegraphics[width=0.9\linewidth]{../results/j2/hohmann/CICICIC_z+.png}
        \caption{View from z+ axis.}
    \end{subfigure}
    \caption{Circle to circle \texttt{CICICIC} maneuver 3D view and projections}
    \label{fig:j2_c2c_CICICIC_figs}
\end{figure}

\begin{figure}[htbp]
    \centering
    \includegraphics[width=0.9\linewidth]{../results/j2/hohmann/CICICIC_primer_vector.png}
    \caption{Primer vector trajectory for Keplerian circle to circle \texttt{CICICIC} rendez-vous.}
    \label{fig:j2_c2c_CICICIC_pv}
\end{figure}

Finally, all cases are summarized in Table~\ref{tab:J2_ctcr_summary}. The final cost is quite close to the Keplerian model cost, but the cost of the optimal maneuver type under the Keplerian model, the \texttt{ICI} maneuver, is wildly different. And again, the primer vector meta-algorithm resulted in a monotonically decreasing sequence of costs.

% J2 C2C SUM
\begin{table}[htbp]
    \centering
    \begin{tabular}{ccc} \toprule
        Maneuver type & Number of impulses & Cost (m/s) \\ \midrule
        \texttt{ICI} & 2 & 9528.19582 \\
        \texttt{CICIC} & 2 & 911.93302 \\
        \texttt{CICICIC} & 3 & 893.05336 \\\bottomrule
    \end{tabular}
    \caption{Summary of J2 Circle to circle RV maneuvers and costs.}
    \label{tab:J2_ctcr_summary}
\end{table}
% J2 C2C SUM

\subsection{Noncoplanar rendez-vous}

Now, the noncoplanar rendez-vous scenario is tackled under the J2 model. As usual, the first maneuver to be solved is a \texttt{ICI} case. The maneuver summary is given in Table~\ref{tab:J2_nr_ICI_tab}, spatial views in Figure~\ref{fig:j2_ncop_ICI_figs}, and the primer vector trajectory in Figure~\ref{fig:j2_ncop_ICI_pv}. Under this model, the non-planarity of the J2 orbits allow for a feasible \texttt{ICI} maneuver with multiple revolutions, much cheaper than the equivalent two-body maneuver. The spatial views show that this maneuver requires a plane change into an intermediate orbit, which intersects the initial and final positions, and a final plane change into the final orbit. The maneuver is still very expensive, and does not satisfy primer vector necessary conditions. In particular, primer vector norm derivative is not continuous. This violation of the necessary conditions does not correspond to any maneuver changes presented in the literature, but previous experience shows that adding initial and final coasts acn greatly reduce the cost. 

% J2 NCOP ICI
\begin{table}[htpb]
    \centering
    \begin{tabular}{cccc} \toprule
    \multicolumn{2}{c}{\textbf{Maneuver type}} & \multicolumn{2}{c}{\texttt{ICI}} \\ \midrule
    \(L\) (m) & \(T\) (s) & \(\varepsilon\) & \(\lVert \Delta \pos_{f} \rVert\) (m)    \\ \midrule
    6.7631e6          & 1.0          & 1.00e-06                & 3.63678e+00                        \\ \midrule
    \(\max \lVert p \rVert\) & 65.441     & \textbf{Diagnostic}   & Add impulse        \\ \midrule
    \textbf{Impulse} & \(t\) (s) & \multicolumn{2}{c}{\(\Delta v\) (m/s)} \\ \midrule
    1                 & 0.0          & \multicolumn{2}{c}{743.66668}                  \\
    2                 & 11107.1576          & \multicolumn{2}{c}{727.80415}                  \\\midrule
    \textbf{Total}   & 11107.1576          & \multicolumn{2}{c}{1471.47082}\\ \bottomrule   
    \end{tabular}
    \caption{Summary of optimization for \texttt{ICI} J2 noncoplanar rendez-vous.}
    \label{tab:J2_nr_ICI_tab}
\end{table}
% J2 NCOP ICI

\begin{figure}[htbp]
    \centering
    \begin{subfigure}{0.49\linewidth}
        \includegraphics[width=0.9\linewidth]{../results/j2/ipv_noncop/ICI_3d.png}
        \caption{3D view.}
    \end{subfigure}
    \begin{subfigure}{0.49\linewidth}
        \includegraphics[width=0.9\linewidth]{../results/j2/ipv_noncop/ICI_x+.png}
        \caption{View from x+ axis.}
    \end{subfigure}
    \begin{subfigure}{0.49\linewidth}
        \includegraphics[width=0.9\linewidth]{../results/j2/ipv_noncop/ICI_y-.png}
        \caption{View from y- axis.}
    \end{subfigure}
    \begin{subfigure}{0.49\linewidth}
        \includegraphics[width=0.9\linewidth]{../results/j2/ipv_noncop/ICI_z+.png}
        \caption{View from z+ axis.}
    \end{subfigure}
    \caption{Noncoplanar rendez-vous \texttt{ICI} maneuver 3D view and projections}
    \label{fig:j2_ncop_ICI_figs}
\end{figure}

\begin{figure}[htbp]
    \centering
    \includegraphics[width=0.9\linewidth]{../results/j2/ipv_noncop/ICI_primer_vector.png}
    \caption{Primer vector trajectory for J2 noncoplanar \texttt{ICI} rendez-vous.}
    \label{fig:j2_ncop_ICI_pv}
\end{figure}
\FloatBarrier
Next, the \texttt{CICIC} maneuver is summarized in Table~\ref{tab:J2_nr_CICIC_tab}, its spatial views are shown in Figure~\ref{fig:j2_ncop_CICIC_figs}, and its primer vector trajectory is shown in Figure~\ref{fig:j2_ncop_CICIC_pv}. Now, this maneuver is only \(5 m/s\) more expensive than its Keplerian counterpart (compare with Table~\ref{tab:Keplerian_nr_CICIC_tab}), but the impulse magnitudes and times are considerably different. The spatial trajectory does not deviate far from the initial and final orbits, sign of small changes in velocity. The primer vector trajectory is now continuous, but still violates the necessary conditions. Now, adding an impulse is necessary.

% J2 NCOP 2 CICIC
\begin{table}[htpb]
    \centering
    \begin{tabular}{cccc} \toprule
    \multicolumn{2}{c}{\textbf{Maneuver type}} & \multicolumn{2}{c}{\texttt{CICIC}} \\ \midrule
    \(L\) (m) & \(T\) (s) & \(\varepsilon\) & \(\lVert \Delta \pos_{f} \rVert\) (m)    \\ \midrule
    6.7631e6          & 11107.158          & 1.00e-06                & 4.12894e-05                        \\ \midrule
    \(\max \lVert p \rVert\) & 3.8353     & \textbf{Diagnostic}   & Add impulse        \\ \midrule
    \textbf{Impulse} & \(t\) (s) & \multicolumn{2}{c}{\(\Delta v\) (m/s)} \\ \midrule
    1                 & 7080.30484          & \multicolumn{2}{c}{24.12503}                  \\
    2                 & 10011.10872          & \multicolumn{2}{c}{34.14936}                  \\\midrule
    \textbf{Total}   & 11107.1576          & \multicolumn{2}{c}{58.27439}\\ \bottomrule   
    \end{tabular}
    \caption{Summary of optimization for \texttt{CICIC} J2 noncoplanar rendez-vous.}
    \label{tab:J2_nr_CICIC_tab}
\end{table}
% J2 NCOP 2 CICIC

\begin{figure}[htbp]
    \centering
    \begin{subfigure}{0.49\linewidth}
        \includegraphics[width=0.9\linewidth]{../results/j2/ipv_noncop/CICIC_3d.png}
        \caption{3D view.}
    \end{subfigure}
    \begin{subfigure}{0.49\linewidth}
        \includegraphics[width=0.9\linewidth]{../results/j2/ipv_noncop/CICIC_x+.png}
        \caption{View from x+ axis.}
    \end{subfigure}
    \begin{subfigure}{0.49\linewidth}
        \includegraphics[width=0.9\linewidth]{../results/j2/ipv_noncop/CICIC_y-.png}
        \caption{View from y- axis.}
    \end{subfigure}
    \begin{subfigure}{0.49\linewidth}
        \includegraphics[width=0.9\linewidth]{../results/j2/ipv_noncop/CICIC_z+.png}
        \caption{View from z+ axis.}
    \end{subfigure}
    \caption{Circle to circle \texttt{CICIC} maneuver 3D view and projections}
    \label{fig:j2_ncop_CICIC_figs}
\end{figure}

\begin{figure}[htbp]
    \centering
    \includegraphics[width=0.9\linewidth]{../results/j2/ipv_noncop/CICIC_primer_vector.png}
    \caption{Primer vector trajectory for J2 circle to circle \texttt{CICIC} rendez-vous.}
    \label{fig:j2_ncop_CICIC_pv}
\end{figure}
\FloatBarrier
A \texttt{CICICIC} maneuver was solved next, with summary in Table~\ref{tab:J2_nr_CICICIC_tab}, primer vector trajectory in Figure~\ref{fig:j2_ncop_CICICIC_pv} and the spatial views are omitted due to the trajectory being visually indistinct to initial and final orbits. The cost is marginally lower than the previous case, but now the primer vector necessary conditions are satisfied. It is worth clarifying that they require that \(\lVert \mathbf{p} \rVert = 1\) at every impulse time, but this does not mean that unity norm cannot occur outside impulse times, which exactly what happens at around \(4000 s\) in this case. 

% J2 NCOP 3 CICICIC
\begin{table}[htpb]
    \centering
    \begin{tabular}{cccc} \toprule
    \multicolumn{2}{c}{\textbf{Maneuver type}} & \multicolumn{2}{c}{\texttt{CICICIC}} \\ \midrule
    \(L\) (m) & \(T\) (s) & \(\varepsilon\) & \(\lVert \Delta \pos_{f} \rVert\) (m)    \\ \midrule
    6.7631e6          & 11107.158          & 1.00e-05                & 1.75767e-07                        \\ \midrule
    \(\max \lVert p \rVert\) & 1.0     & \textbf{Diagnostic}   & Final coast        \\ \midrule
    \textbf{Impulse} & \(t\) (s) & \multicolumn{2}{c}{\(\Delta v\) (m/s)} \\ \midrule
    1                 & 1676.61473          & \multicolumn{2}{c}{5.84342}                  \\
    2                 & 7185.69293          & \multicolumn{2}{c}{20.83452}                  \\
    3                 & 9942.01138          & \multicolumn{2}{c}{29.32859}                  \\\midrule
    \textbf{Total}   & 11107.1576          & \multicolumn{2}{c}{56.00653}\\ \bottomrule   
    \end{tabular}
    \caption{Summary of optimization for \texttt{CICICIC} J2 noncoplanar rendez-vous.}
    \label{tab:J2_nr_CICICIC_tab}
\end{table}
% J2 NCOP 3 CICICIC

\begin{figure}[htbp]
    \centering
    \includegraphics[width=0.9\linewidth]{../results/j2/ipv_noncop/CICICIC_primer_vector.png}
    \caption{Primer vector trajectory for J2 circle to circle \texttt{CICICIC} rendez-vous.}
    \label{fig:j2_ncop_CICICIC_pv}
\end{figure}

Finally, all results are summarized in Table~\ref{tab:J2_nr_summary}. The optimal maneuver now has one impulse less than the corresponding Keplerian maneuver, and a significantly higher cost.

% J2 NCOP SUM
\begin{table}[htbp]
    \centering
    \begin{tabular}{ccc} \toprule
        Maneuver type & Number of impulses & Cost (m/s) \\ \midrule
        \texttt{ICI} & 2 & 1471.47082 \\
        \texttt{CICIC} & 2 & 58.27439 \\
        \texttt{CICICIC} & 3 & 56.00653 \\\bottomrule
    \end{tabular}
    \caption{Summary of J2 noncoplanar rendez-vous maneuvers and costs.}
    \label{tab:J2_nr_summary}
\end{table}
% J2 NCOP SUM

\newpage
\FloatBarrier
\section{J2 and Drag}

Finally, the non-conservative model of J2 and drag disturbances is tackled. This model's results will be presented in a summarized fashion, since they were found to be almost identical to J2 results. In addition, although the theory predicts that only the ODE method of primer vector calculation should work for this system, it was found that the STM method gave almost identical trajectories. Both of those findings will be discussed based on the \texttt{ICI} maneuver of the circle-to-circle scenario.

\subsection{Circle to Circle}

The circle to circle case was solved in a \texttt{ICI} maneuver case, summarized in Table~\ref{tab:jd_c2c_ICI_tab}. Straightaway, it is clear that these results are numerically identical to the ones found in the J2 model, given in Table~\ref{tab:J2_nr_ICI_tab}. This can be easily explained by the fact that the effects of drag are some orders of magnitude weaker than J2 effects and, above all, they do not change the trajectories categorically like J2 does. That is, J2 makes trajectories noncoplanar, but drag does not introduce any particular behavior that is noticeable in short time spans like this.
    
\begin{table}[htpb]
    \centering
    \begin{tabular}{cccc} \toprule
    \multicolumn{2}{c}{\textbf{Maneuver type}} & \multicolumn{2}{c}{ICI} \\ \midrule
    \(L\) (m) & \(T\) (s) & \(\varepsilon\) & \(\Delta x_{f}\) (m)    \\ \midrule
    8.0e6          & 1.0          & 1.00e-05                & 0.49712                        \\ \midrule
    \(\max \lVert p \rVert\) & 563.46     & \textbf{Diagnostic}   & Initial + Final coast        \\ \midrule
    \textbf{Impulse} & \(t\) (s) & \multicolumn{2}{c}{\(\Delta v\) (m/s)} \\ \midrule
    1                 & 0.0          & \multicolumn{2}{c}{5209.4779}                  \\
    2                 & 3560.54079          & \multicolumn{2}{c}{4318.71793}                  \\\midrule
    \textbf{Total}   & 3560.54079          & \multicolumn{2}{c}{9528.19583}\\ \bottomrule   
    \end{tabular}
    \caption{Summary of optimization for J2+Drag circle-to-circle \texttt{ICI} rendez-vous.}
    \label{tab:jd_c2c_ICI_tab}
\end{table}

The primer vector trajectory in Figure~\ref{fig:jd_c2c_ICI_pv} compares the STM and ODE methods for this maneuver. As a reminder, the theory from Section~\ref{sssec:pv_calc_ncon} disallows, in general, the use of the STM method with non-conservative models. However, as can be seen, both methods give almost identical primer vector trajectories.

\begin{figure}[htbp]
    \centering
    \includegraphics[width=\textwidth]{../results/j2drag/hohmann/ICI_primer_vector.png}
    \caption{Primer vector trajectory for J2+Drag circle to circle \texttt{ICI} rendez-vous.}
    \label{fig:jd_c2c_ICI_pv}
\end{figure}

This can be explained by comparing the matrices \(\mathbf{A}_\delta\) and \(\mathbf{A}_p\), the coefficients of the ODEs of the STM and PVTM respectively. They are only expected to differ in the lower right quadrant, shown here for the initial condition:
\begin{align}
    (\mathbf{A}_\delta)_{4\leq i \leq 6, 4 \leq j \leq 6} &= \begin{bmatrix}
        -5.56731-12 & 0.0 & 0.0 \\
        0.0 & -7.77221e-12    & -2.72282e-12 \\
        0.0 & -2.72282e-12    & -8.92972e-12 \\
    \end{bmatrix} \\
    (A_p)_{4\leq i \leq 6, 4 \leq j \leq 6} &= \begin{bmatrix}
        5.56731e-12 & 0.0 & 0.0 \\
        0.0 & 7.77221e-12 & 2.72282e-12 \\
        0.0 & 2.72282e-12 & 8.92972e-12 \\
    \end{bmatrix}
\end{align}

Indeed, the matrices satisfy \((\mathbf{A}_\delta)_{4\leq i \leq 6, 4 \leq j \leq 6} = - (A_p)_{4\leq i \leq 6, 4 \leq j \leq 6}^T\), as predicted by the theory in Section~\ref{sssec:pv_calc_ncon}, but due to the low intensity of drag forces, they are both close to \(\mathbf{0}_{3\times 3}\), which is the lower right quadrant of \(\mathbf{A}_{p, J2}\), the primer vector ODE coefficient matrix in the J2 model. This shows that maneuvers where drag is weak, either by virtue of the short timespan or high altitude, the effects of drag may be neglected. An open question is whether or not there exists a realistic maneuver scenario where drag effects cause sufficient differences with respect to to the J2 case.

%J2 Drag C2C SUM
\begin{table}[htbp]
    \centering
    \begin{tabular}{ccc} \toprule
        Maneuver type & Number of impulses & Cost (m/s) \\ \midrule
        \texttt{ICI} & 2 & 9528.19583 \\
        \texttt{CICIC} & 2 & 911.93304 \\
        \texttt{CICICIC} & 3 & 893.05339 \\\bottomrule
    \end{tabular}
    \caption{Summary of J2+Drag Circle to circle RV maneuvers and costs.}
    \label{tab:J2_Drag_ctcr_summary}
\end{table}
%J2 Drag C2C SUM

\subsection{Noncoplanar rendez-vous}

%ICI is infeasible

% J2D NCOP ICI
\begin{table}[htpb]
    \centering
    \begin{tabular}{cccc} \toprule
    \multicolumn{2}{c}{\textbf{Maneuver type}} & \multicolumn{2}{c}{\texttt{ICI}} \\ \midrule
    \(L\) (m) & \(T\) (s) & \(\varepsilon\) & \(\lVert \Delta \pos_{f} \rVert\) (m)    \\ \midrule
    6.7631e6          & 1.0          & 1.00e-05                & 1.22605e-07                        \\ \midrule
    \(\max \lVert p \rVert\) & 1.0     & \textbf{Diagnostic}   & Local optimum        \\ \midrule
    \textbf{Impulse} & \(t\) (s) & \multicolumn{2}{c}{\(\Delta v\) (m/s)} \\ \midrule
    1                 & 0.0          & \multicolumn{2}{c}{11743.60995}                  \\
    2                 & 11107.1576          & \multicolumn{2}{c}{11711.00712}                  \\\midrule
    \textbf{Total}   & 11107.1576          & \multicolumn{2}{c}{23454.61707}\\ \bottomrule   
    \end{tabular}
    \caption{Summary of optimization for \texttt{ICI} J2+Drag noncoplanar rendez-vous.}
    \label{tab:J2_Drag_nr_ICI_tab}
\end{table}
% J2D NCOP ICI

% J2D NCOP 2 CICIC
\begin{table}[htpb]
    \centering
    \begin{tabular}{cccc} \toprule
    \multicolumn{2}{c}{\textbf{Maneuver type}} & \multicolumn{2}{c}{\texttt{CICIC}} \\ \midrule
    \(L\) (m) & \(T\) (s) & \(\varepsilon\) & \(\lVert \Delta \pos_{f} \rVert\) (m)    \\ \midrule
    6.7631e6          & 11107.158          & 1.00e-05                & 1.58998e-08                        \\ \midrule
    \(\max \lVert p \rVert\) & 3.849     & \textbf{Diagnostic}   & Add impulse        \\ \midrule
    \textbf{Impulse} & \(t\) (s) & \multicolumn{2}{c}{\(\Delta v\) (m/s)} \\ \midrule
    1                 & 7078.65384          & \multicolumn{2}{c}{24.08948}                  \\
    2                 & 10011.90024          & \multicolumn{2}{c}{34.22663}                  \\\midrule
    \textbf{Total}   & 11107.1576          & \multicolumn{2}{c}{58.31611}\\ \bottomrule   
    \end{tabular}
    \caption{Summary of optimization for \texttt{CICIC} J2+Drag noncoplanar rendez-vous.}
    \label{tab:J2_Drag_nr_CICIC_tab}
\end{table}
% J2D NCOP 2 CICIC

% J2D NCOP 3 CICICIC
\begin{table}[htpb]
    \centering
    \begin{tabular}{cccc} \toprule
    \multicolumn{2}{c}{\textbf{Maneuver type}} & \multicolumn{2}{c}{\texttt{CICICIC}} \\ \midrule
    \(L\) (m) & \(T\) (s) & \(\varepsilon\) & \(\lVert \Delta \pos_{f} \rVert\) (m)    \\ \midrule
    6.7631e6          & 11107.158          & 1.00e-05                & 2.59433e-06                        \\ \midrule
    \(\max \lVert p \rVert\) & 1.0     & \textbf{Diagnostic}   & Final coast        \\ \midrule
    \textbf{Impulse} & \(t\) (s) & \multicolumn{2}{c}{\(\Delta v\) (m/s)} \\ \midrule
    1                 & 1676.62119          & \multicolumn{2}{c}{5.8844}                  \\
    2                 & 7185.71036          & \multicolumn{2}{c}{20.80379}                  \\
    3                 & 9941.99005          & \multicolumn{2}{c}{29.326}                  \\\midrule
    \textbf{Total}   & 11107.1576          & \multicolumn{2}{c}{56.01418}\\ \bottomrule   
    \end{tabular}
    \caption{Summary of optimization for \texttt{CICICIC} J2+Drag noncoplanar rendez-vous.}
    \label{tab:J2_Drag_nr_CICICIC_tab}
\end{table}
% J2D NCOP 3 CICICIC

% J2D NCOP SUM
\begin{table}[htbp]
    \centering
    \begin{tabular}{ccc} \toprule
        Maneuver type & Number of impulses & Cost (m/s) \\ \midrule
        \texttt{ICI} & 2 & 23454.61707 \\
        \texttt{CICIC} & 2 & 58.31611 \\
        \texttt{CICICIC} & 3 & 56.01418 \\\bottomrule
    \end{tabular}
    \caption{Summary of J2+Drag noncoplanar rendez-vous maneuvers and costs.}
    \label{tab:J2_Drag_nr_summary}
\end{table}
% J2D NCOP SUM