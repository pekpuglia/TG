
In this chapter, previous results relating to the field of orbital maneuvering are presented. Relevant research spans the 1960's all the way to recent years. In particular, results about the solution and application of the Lambert problem, usage of primer vector theory and direct optimization approaches are presented.

TODO add Multiple shooting

% \section{Lambert Problem}

% Efficient solutions to the Lambert problem have been of interest for many centuries now, and many algorithms exist, with varying convergence guarantees and indetermination handling. An algorithm based on continued fractions~\cite{battin_vaughan_elegant_lambert} has been proposed, and it shows promising results in terms of number of iterations until convergence. It also relies on an orbital transformation to bypass the indeterminate case, choosing one of many possible solutions. However, it requires previous knowledge of the type of orbit (elliptic, parabolic, hyperbolic) and good initial guesses for parameters with difficult physical interpretation, making it not suitable for trajectory optimization.

% \citeauthor{embedded_lambert} (2020) present a way of applying many embedded Lambert Problems to trajectory optimization. In their approach, time is also an optimization variable. Many orbital segments of equal duration are concatenated and impulses are allowed to happen at each boundary. Then, a Cartesian variables-based Lambert problem is solved for each segment. Impulses at segment boundaries are found to be significant only at a handful of boundaries. With a big number of segments, this approximates the number, instant and magnitudes well. The Cartesian variables-based Lambert problem solution is indeed applicable to trajectory optimization, and this result also offers good insights into how to find appropriate transfer times between orbits.

\section{Primer Vector theory}

The evolution of the primer vector along coasting arcs is of interest for the solution of impulsive maneuvering problems. An analytical form of the state transition matrix is given in~\citeauthor{glandorf_transition_matrix} (1969). A clever choice of time-changing basis allows for the direct solution of the components of the primer vector differential equations (Lagrange multiplier equations), which can then be written in matrix form. The state transition matrix is given as the product of a time-evolving matrix with its inverse at the initial time; closed form expressions are given for the \(6\times6\) state transition matrix. Despite being restricted to the two body problem, this result proves really useful since it allows the circumvention of the solution of a TPBVP.

\citeauthor{fixed_time_primer_vector} (1968) also give a closed-form expression for the state transition matrix and a review of the application of primer vector theory. In particular, several common cases of primer vector trajectories are explored, and the relationship between suboptimal primer vector trajectories and the necessary changes is explored. Overall, Lion and Handelsman provide useful examples for understanding this theory. 

Finally, an interactive algorithm for computing the optimal number of impulses based on analyzing the primer vector trajectories is given in \citeauthor{interactive_primer_vector} (2010). The discrete nature of the variable ``number of impulses'' requires tools more powerful than continuous variable nonlinear solvers, such as evolutionary algorithms. The proposed algorithm iterates through proposing an \(n\) impulse maneuver, analyzing its primer vector history and optimizing impulse times, if needed, and then adding maneuvers if the necessary conditions are not yet satisfied. This remains one of the most direct ways of optimizing the number of impulses; however, the usage of primer vector theory provides only \textit{necessary} conditions, which can lead to many false (or local) optima. Similar approaches are found in \citeauthor{efficient_n_impulse} (1968), which applies it to an Apollo rendezvous, with a large plane change maneuver.

TODO ADD RYAN RUSSELL, BRUNO

\section{Direct Optimization}

The central question of how many impulses are needed for a maneuver is tackled through the continuous-thrust approach in \citeauthor{how_many_impulses} (2019). A continuous thrust model is established and a sequence of increasing maximum thrust values is analyzed. Starting with the minimum thrust needed to execute the maneuver in the given time (case in which the engines fire incessantly), the thrust is continuously increased, and gaps in the thrusting times appear. In the limiting case of very high thrusts approaching infinity, engine firing closely resembles an impulse. This gives, to very good approximation, the needed number of impulses. The main disadvantage of this approach is the need for a continuous thrust model, in addition to an impulse thrust model. 

A similar approach is presented in \citeauthor{mult_rev_many_imp} (2023), where a ``control sweep'' is performed to identify how many impulses are necessary, thus excluding the number of impulses as a problem variable. In particular, both continuous thrust constant specific impulse problems and continuous thrust variable specific impulse problems (not treated in the present work) are used during this step. These problems help in finding better optima in multi-revolution maneuvers, where multiple local optima exist. In addition, the importance of impulsive solutions is reaffirmed. Despite being abstractions, impulsive solutions provide lower bounds on fuel usages, can provide feasibility insights and can be abstracted from spacecraft mass.

The existence of multiple global optima is explored in \citeauthor{inf_many_optima} (2023), which proposes a method for generating families of transfer orbits with many impulses starting with a seed two-impulse maneuver. From a theoretical point of view, this highlights the non-convexity of the orbital maneuver optimization problem (existence of many optima), often tied to the periodicity of orbits. From a practical point of view, it exposes the concept of \textit{phasing orbit}, an intermediate orbit between two consecutive impulses that can be chosen arbitrarily from certain period values without changing the delta-V budget. Also, the possibility of reducing flight time based on some of those solutions is also explored. However, these results are mostly applicable to long time horizon problems.

In \citeauthor{multi_impulse_circ_rendezvous} (1986), multiple impulse maneuvers for fixed time rendez-vous between spacecraft in a subclass of circular orbits is considered. The fixed time rendez-vous problem is analogous to the problem of transfer to a particular state in fixed time. Despite the restriction to a subset of possible orbits, it is shown that there is a certain minimum time required for the optimal time-open solutions (such as Hohmann transfers) to be found. Again, primer vector theory is applied but sometimes leads to local optima, and divergences from the established primer vector algorithm can sometimes be preferred over following it stricly.



