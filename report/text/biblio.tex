
In this chapter, previous results relating to the field of orbital maneuvering and optimal are presented. Relevant research spans the 1960's all the way to recent years. In particular, some results about maneuver optimization, the multiple shooting method and primer vector applications are laid out in the next sections.

\section{Orbit Propagation}

Precise orbit propagation is desirable in many space applications with long time horizons. Preserving physical quantities like energy and angular momentum are not guaranteed with traditional integration method. Sympletic integrators are proposed for orbital propagation in \citeonline{sympletic_int}, which are designed so as to preserve physical quantities about the problem. They can greatly increase precision in integration for three-body and two-body problems. Integrators can be further divided into explicit or implicit integrators, dependending on whether or not the differential equation can be propagated sequentially forwards in time or not. Another characteristic of orbital propagators that may be desired is speed of integration. Again, in application with long time horizons, propagation of a satellite's trajectory may take a considerable amount of time. Some integrators are amenable to parallel computation~\cite{implicit_rk}, leading to great saving in computation time.\

In the class of explicit integrators, orbital mechanics greatly profit from Runge Kutta (RK) methods of high order~\cite{num_int_orb}. The oscillatory nature of orbits and the long time scales involved are better approximated by these methods than by low (4 and below) order methods. A suite of Runge Kutta methods with the necessary coefficients is given in~\citeonline{rk8}. These coefficients are used, for instance, in MATLAB's \texttt{odeN(N+1)} integrator suite. An important figure of merit for high order integrators is the number of necessary function calls to the function describing the system's dynamics. The RK4 method requires 4 function calls, and Verner's RK8 method requires 16. A RK8 method with 11 function calls has been described~\cite{rk8_11calss}, which was later proven to be the best possible number of calls~\cite{non_existence_rk8_10calss}.

Another entirely different class of orbital propagators is that of semi-analytical propagators, that rely on Kepler's Equation and Gauss' variational equations to propagate orbital elements. A classical example of this is the SGP4 propagator~\cite{spacetrack_report_3_revisit}, one of the earliest publicly available orbital propagation tools including some orbital perturbations. It is still the subject of research, with recent improvements in its precision with machine learning techniques and differentiable programming~\cite{sgp4_high_precision}. 

A recent addition to the publicly available orbit propagators is the SatelliteToolbox, a Brazilian Julia library that implements tested orbital propagator for many different models, including SGP4~\cite{satellitetoolbox}. Although no new algorithm is proposed, its implementation provides insights into modern numerical techniques for fast propagation of long missions, such as the Amazonia 1 mission. In particular, strong typing and re-use of previous calculations allow for stable and fast propagation under simplified models suitable for Earth orbit missions.\

Many works on finite thrust maneuvering rely on modified equinocial elements (MEEs), which are an orbital state parameterization with no singularities and relatively simple time derivatives when subject to perturbations~\cite{mod_equinoctial_els}. They also benefit from having one fast-changing state, the true longitude, and many slow-changing states, allowing for simpler integration schemes and more stable numerical properties compared to Cartesian coordinates~\cite{Conway_2010}. They are used for instance in \citeonline{low_thrust_equinoctial_els}, where a thrust continuation method is used to approach impulsive solutions with finite thrust. For impulsive maneuvers, \citeonline{how_many_impulses} have applied them successfully to the optimization of multiple-revolution transfers. MEEs lead. They are not ubiquitous, however, and a Cartesian parameterization of the state is also common, as in \citeonline{russell_primer_low_thrust}, particularly in three-body problems. Despite the numerical disadvantages, Cartesian elements are simple and applicable to the scope of this work.

\section{Maneuver Optimization}

The central question of how many impulses are needed for a maneuver is tackled through the continuous-thrust approach in \citeonline{how_many_impulses}. A continuous thrust model is established and a sequence of increasing maximum thrust values is analyzed. Starting with the minimum thrust needed to execute the maneuver in the given time (case in which the engines fire incessantly), the thrust is continuously increased, and gaps in the thrusting times appear. In the limiting case of very high thrusts approaching infinity, engine firing closely resembles an impulse. This gives, to very good approximation, the needed number of impulses. The main disadvantage of this approach is the need for a continuous thrust model, in addition to an impulse thrust model. 

A similar approach is presented in \citeonline{mult_rev_many_imp}, where a ``control sweep'' is performed to identify how many impulses are necessary, thus excluding the number of impulses as a problem variable. This is an approach known as \textit{homotopic continuation} in optimal control. In particular, both continuous thrust constant specific impulse problems and continuous thrust variable specific impulse problems (not treated in the present work) are used during this step. These problems help in finding better optima in multi-revolution maneuvers, where multiple local optima exist. In addition, the importance of impulsive solutions is reaffirmed. Despite being abstractions, impulsive solutions provide lower bounds on fuel usages, can provide feasibility insights and can be abstracted from spacecraft mass.

The existence of multiple global optima is explored in \citeonline{inf_many_optima}, which proposes a method for generating families of transfer orbits with many impulses starting with a seed two-impulse maneuver. From a theoretical point of view, this highlights the non-convexity of the orbital maneuver optimization problem (existence of many optima), often tied to the periodicity of orbits. From a practical point of view, it exposes the concept of \textit{phasing orbit}, an intermediate orbit between two consecutive impulses that can be chosen arbitrarily from certain period values without changing the delta-V budget. Also, the possibility of reducing flight time based on some of those solutions is also explored. However, these results are mostly applicable to long time horizon problems.

In \citeonline{multi_impulse_circ_rendezvous}, multiple impulse maneuvers for fixed time rendez-vous between spacecraft in a subclass of circular orbits is considered. The fixed time rendez-vous problem is analogous to the problem of transfer to a particular state in fixed time. Despite the restriction to a subset of possible orbits, it is shown that there is a certain minimum time required for the optimal time-open solutions (such as Hohmann transfers) to be found. Again, primer vector theory is applied but sometimes leads to local optima, and divergences from the established primer vector algorithm can sometimes be preferred over following it stricly.

\section{Primer Vector theory}

The evolution of the primer vector along coasting arcs is of interest for the solution of impulsive maneuvering problems. An analytical form of the state transition matrix is given in~\citeonline{glandorf_transition_matrix}. A clever choice of time-changing basis allows for the direct solution of the components of the primer vector differential equations (Lagrange multiplier equations), which can then be written in matrix form. The state transition matrix is given as the product of a time-evolving matrix with its inverse at the initial time; closed form expressions are given for the \(6\times6\) state transition matrix. Despite being restricted to the two body problem, this result proves really useful since it allows the circumvention of the solution of a TPBVP.

\citeonline{fixed_time_primer_vector} also give a closed-form expression for the state transition matrix and a review of the application of primer vector theory. In particular, several common cases of primer vector trajectories are explored, and the relationship between suboptimal primer vector trajectories and the necessary changes is explored. Overall, Lion and Handelsman provide useful examples for understanding this theory. 

Finally, an interactive algorithm for computing the optimal number of impulses based on analyzing the primer vector trajectories is given in \citeonline{interactive_primer_vector}. The discrete nature of the variable ``number of impulses'' requires tools more powerful than continuous variable nonlinear solvers, such as evolutionary algorithms. The proposed algorithm iterates through proposing an \(n\) impulse maneuver, analyzing its primer vector history and optimizing impulse times, if needed, and then adding maneuvers if the necessary conditions are not yet satisfied. This remains one of the most direct ways of optimizing the number of impulses; however, the usage of primer vector theory provides only \textit{necessary} conditions, which can lead to many false (or local) optima. Similar approaches are found in \citeonline{efficient_n_impulse} (1968), which applies it to an Apollo rendezvous, with a large plane change maneuver.

Primer vector theory has also been applied outside the scope of Earth orbit. In \citeonline{bruno_asteroid_primer}, primer vector theory is used successfully for the optimization of an asteroid rendez-vous mission, with an adapted cost function. In \citeonline{impulsive_europa}, the primer vector is applied in a multibody-environment, with the STM method used in this work being introduced. However, no instances of primer vector theory applied to non-conservative systems has been found, which is the gap this work aims to fill.