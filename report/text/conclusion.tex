In summary, this work set out to optimize impulsive maneuvers in LEO, apply primer vector theory to the optimized maneuvers and adapt this theory to the types of perturbations encountered in the space environment around Earth. The main contribution of this work was to extend primer vector theory to perturbed models and propose primer vector calculation methods that are valid under different perturbation classes. An impulsive multiple shooting optimization scheme in Cartesian coordinates was successfully implemented with the CasADi library, and validated under Keplerian dynamics for the well known Hohmann transfer case, and for a more complex noncoplanar rendez-vous scenario proposed in the primer vector literature. The need for variable scaling was made evident even in the simplest maneuver cases in this work. The most remarkable characteristic of Keplerian maneuvers is the fact that they are composed of piecewise planar segments, which greatly restricts which states are accessible from any impulse position. The existence of many local optima was verified in practice by comparison of this work's results with previous work available in the literature, highlighting the importance of sampling initial guesses. The partition of time between coasting arcs was found to be the most significant initial guess parameter, altering the positions of impulses and therefor, their effect. This was exploited to great effect in a simple initial guess resampling scheme. In both scenarios, the Glandorf, STM and ODE primer vector calculation methods were proposed and validated between each other as means to compute the primer vector transition matrix.

The modelling of Earth's nonspherical gravity field with the J2 model was found to greatly alter the maneuver trajectories, even in short time frames. J2 trajectories are not planar, which changes categorically which states are reachable from other states. Impulse instants, magnitudes and directions as well as the number thereof can change under the J2 model when compared with the Keplerian model, highlighting the importance of modelling orbital perturbations in LEO\@. This modelling excludes the Glandorf method from being used, as expected from the theory, which was numerically verified by comparison between the Glandorf and STM, or ODE, methods. As a conservative model, STM and ODE primer vector calculations were expected to be identical, which was confirmed in practice.

Finally, the modelling of a non-conservative perturbation such as drag introduced yet another layer of differences with respect to the Keplerian model ubiquitous in the orbital maneuvering literature. The commonly used piecewise models for atmospheric density had to be smoothened for compatibility with Ipopt, the nonlinear optimizer used. Primer vector theory under this type of model was not found in the literature, and an extension has been proposed in this work. The primer vector differential equations were derived considering an acceleration term that depends on position and velocity, and the rest of the theory was adapted based on this. The STM method was predicted to not be applicable, with the ODE method being the only valid method for such a general class of models. In practice, however, drag is a very weak perturbation of some orders of magnitude less intensity than the J2 force. In the scenarios considered, the drag perturbation added slight differences in trajectory and cost, but nothing as significant as the modelling of J2 forces. In addition, the STM method was found to be indistinguishable from the theoretically sound ODE method, and both resulted in primer vector trajectories very similar to those under the J2 model. Both of those effects are conjectured to be linked: should there be a maneuver scenario where J2 and J2+Drag solutions are significantly diffrent, then the STM and ODE methods should differ. Due to the need for drag smoothing, CasADi model sizes were significantly bigger than in previous cases.

Future works might investigate under which conditions the effects of drag alter orbital maneuvers and the primer vector trajectories, validating or refuting the conjectured relation between both of those alterations. Long horizon scenarios, as well as aerobraking maneuvers, can be considered as means for testing this. In addition, performance gains could be searched by reducing CasADi model sizes or changing the problem's parameterization from Cartesian to modified equinoctial elements. Lastly, no general method for estimating lower bounds of maneuver costs is known; the methods presented in this work would greatly profit from a lower bound estimation. The primer vector algorithms offer a monotonically decreasing sequence of costs, and the knowledge of a lower bound on the global cost could greatly improve confidence in the results and rigor in the field.