In summary, this work set out to apply primer vector theory to optimal impulse maneuvers in LEO and adapt it to the types of perturbations encountered in this environment. An impulsive multiple shooting optimization scheme in Cartesian coordinates was successfully implemented, and validated under Keplerian dynamics for the well known Hohmann transfer case, and for a more complex noncoplanar rendez-vous scenario proposed in the primer vector literature. The same scenarios were solved under conservative and non-conservative perturbation models, allowing for the comparison of the effect of perturbations in the optimal maneuvers and in the primer vector algorithm. For each model, some valid methods of primer vector calculation have been tried and validated between each other, and the primer vector has proven to be a useful tool in reducing the cost of orbital maneuvers. As for the perturbations, the modelling of the J2 force greatly alters the nature of orbital maneuvers, mainly due to the fact that orbits are no longer planar. The addition of drag in the orbital model does not alter significantly the optimal maneuvers or the primer vector analysis, which allows the usage of the STM method for its calculation, even though in general it is predicted that the STM method will not work for non-conservative models. 

Future works might investigate under which conditions the effects of drag alter orbital maneuvers and the primer vector trajectories, as well as increase the time horizon of the maneuvers considered. In addition, the exploration of possible performance gains from using modified equinoctial elements instead of a Cartesian formulation might be investigated. Lastly, no general method for estimating lower bounds of maneuver costs is known; the methods presented in this work would greatly profit from a lower bound estimation.

TODO checklist, 

TODO agradecimentos

TODO reler

TODOformatação