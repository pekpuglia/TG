\section{Maneuvers of Interest}

Some orbital maneuvers are chosen as representative of general space mission needs for comparing models and solvers. Maneuvers are presented in table~\ref{tab:man_interest}. 

\begin{table}[htpb]
    \centering
    \begin{tabular}{>{\centering\arraybackslash}m{3cm}>{\centering\arraybackslash}m{3cm}>{\centering\arraybackslash}m{3cm}>{\centering\arraybackslash}m{3cm}}\toprule
        \textbf{Name/ application} & \textbf{Initial condition} & \textbf{Final condition} & \textbf{Comment}\\ \midrule
        LEO maintenance & Circular orbit at \(h \approx 400km\) & Circular orbit at \(h \approx 600km\) & Coplanar, (equatorial: reproduce Hohmann?, inclined: reproduce \cite{sandro_quasi_circ}) \\ \midrule
        SSO injection & Elliptic \(200/800/98^\circ\) & Circular \(750km\), \(98^\circ \) & Coplanar or small \(\Omega\)/\(i\) correction? \\ \midrule
        Constellation ideshare injection & Circular \(800km\) equatorial & Circular \(600km\), \(i=30^\circ\) & Target point should move according to J2 model? \\ \midrule
        LEO to GEO transfer & Circular \(600km\) & Circular \(35000km\) & Explore optimal transfer with multiple impulses, add small plane change correction? \\ \midrule
        Escape velocity (with?) Moon & Circular \(600km\), equatorial & Target velocity vector \(V(t \rightarrow \infty) = \vec{V}_e\) & Variant: start in eliptical orbit (verify perigee is more efficient). Goal: use Moon for gravity assist \\ \bottomrule
    \end{tabular}
    \caption{List of orbital maneuvers of interest.}\label{tab:man_interest}
\end{table}

\section{Minimum Viable Thesis (MVT)}

Impulsive solutions have coasting segments separated by instant impulses. Given a coasting model for the free evolution of orbital parameters \(\Theta\):
\begin{equation} \label{eq:coasting}
    \Theta(t) = f(\Theta_0, t_0, t)
\end{equation}

Solve a parameter optimization problem for a list of \(N\) impulses parameterized by:
\begin{equation} 
    p_{\text{impulse}i} = \begin{bmatrix}
        t_{\text{impulse},i} \\
        \Delta \vec{V}_i
    \end{bmatrix},
\end{equation}
which configures a sequence of continuous piecewise trajectories described by~\eqref{eq:coasting},

Taking a spacecraft from an initial given condition to a final condition in \textbf{finite time} while maximizing final mass
\begin{equation} \label{eq:max_m}
    \max_{T(t)} m_f 
\end{equation}
which, for constant specific impulse engines, is equivalent to minimize the total acceleration or the sum of \(\Delta V\)'s:
\begin{equation} \label{eq:min_dV}
    \min \int_{t_0}^{t_f} \Gamma(t) dt = \min \sum_{i=1}^N \Delta \vec{V}_i
\end{equation}

Then, verify optimality of the solution by applying primer vector theory as in \cite{Conway_2010} chapter 2 and find optimal number of impulses.

\subsection{Comments}

\begin{itemize}
    \item Model~\eqref{eq:coasting} is supplied by Julia's Satellite Toolbox, for example J2 analytical orbit propagator
    \item Which objective function,~\eqref{eq:max_m} or~\eqref{eq:min_dV} is more advantageous?
    \item 
\end{itemize}

\section{Finite Thrust Dynamical Transfer}

The system is described by a dynamical equation
\begin{equation} \label{eq:dynamics}
    \dot \Theta = f(\Theta, u, t)
\end{equation}
where the thrust input is constrained \(0 \leq \lVert u \rVert \leq T_{\max}\). Solve for a control history \(u(t)\) taking the system from a given initial position to a given final position in finite time while maximizing final mass as in~\eqref{eq:max_m}. This objective has 2 versions depending on engine model shown in table~\ref{tab:prop_models}~\cite{Conway_2010}.

\begin{table}[]
    \centering
    \begin{tabular}{p{5cm}p{5cm}p{5cm}} \toprule
        & \textbf{Constant Specific Impulse} & \textbf{Variable Specific Impulse} \\ \midrule
        Description & Thrust limited, usually chemical engine & Power limited, usually electric engine \\
        Objective Function corresponding to~\eqref{eq:max_m} & \(\min \int_{t_0}^{t_f} \Gamma(t) dt\) & \(\min \int_{t_0}^{t_f} \Gamma^2(t) dt\) \\ \bottomrule
    \end{tabular}
    \caption{Propulsion models and objective functions}
    \label{tab:prop_models}
\end{table}

Many dynamical models of the form~\eqref{eq:dynamics} exist, taking into account different physical phenomena and different parameterizations. This, combined with the multiple solver options, makes for a very repetitive combinatorical problem. 

\textbf{TODO}
\begin{itemize}
    \item decide validation model (capderou)
    \item decide model x solver x problem
\end{itemize}