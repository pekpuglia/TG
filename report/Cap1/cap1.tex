\section{Method of Validation}

All generated maneuvers are to be tested with the full model using SatelliteToolbox, considering atmospheric \& sun \& moon models + non spherical gravity model.

All control solutions are to be analyzed with primer vector theory to verify satisfaction of optimality necessary conditions.

\section{Problem Statement}

The system is described by a dynamical equation
\begin{equation} \label{eq:dynamics}
    \dot \Theta = f(\Theta, u, t)
\end{equation}
on the orbital parameter vector \(\Theta\). The goal is taking the satellite from an initial condition \(\Theta(0) = \Theta_0\) to a final condition specified by a constraint \(\Psi(\Theta(t_f)) = 0\) in finite time \(t_f\) while maximizing final mass:

\begin{equation} \label{eq:max_m}
    \max_{T(t)} m_f.
\end{equation}

\section{Maneuvers of Interest}

Some orbital maneuvers are chosen as representative of general space mission needs for comparing models and solvers. Maneuvers are presented in table~\ref{tab:man_interest}. 

\begin{table}[htpb]
    \centering
    \begin{tabular}{>{\centering\arraybackslash}m{3cm}>{\centering\arraybackslash}m{3cm}>{\centering\arraybackslash}m{3cm}>{\centering\arraybackslash}m{3cm}}\toprule
        \textbf{Name/ application} & \textbf{Initial condition} & \textbf{Final condition} & \textbf{Comment}\\ \midrule
        LEO maintenance & Circular orbit at \(h \approx 400km\) & 
        Circular orbit at \(h \approx 600km\) & Coplanar, equatorial reproduce Hohmann \\ \midrule

        SSO maintenance & Circular \(700km\), \(i=95^\circ\), \(\Delta Omega = 5^\circ\) & Circular \(750km\), \(98^\circ \) & Small correction in multiple orbital elements \\ \midrule
        
        Constellation rideshare phasing & Circular \(800km\)equatorial & Circular \(800km\), \(\Delta \Phi = 30^\circ\) & Coplanar, inclined, consider J2 \\ \midrule
        
        LEO to GEO transfer & Circular \(600km\) & Circular \(35000km\) & Explore optimal transfer with multiple impulses, add small plane change correction? \\ \midrule
        
        \textbf{Optional} Escape velocity (with?) Moon & Circular \(600km\), equatorial & Target velocity vector \(V(t \rightarrow \infty) = \vec{V}_e\) & Variant: start in eliptical orbit (verify perigee is more efficient). Goal: use Moon for gravity assist \\ \bottomrule
    \end{tabular}
    \caption{List of orbital maneuvers of interest.}\label{tab:man_interest}
\end{table}

Each maneuver in this set needs to be combined with a propulsion model, a propagation model, and a solver. Options for each are discussed in the following sections.

\newpage
\section{Propulsion model}

Rocket engine models are exposed in table~\ref{tab:prop_models}. Objective function comes from~\cite{Conway_2010} and~\cite{sandro_quasi_circ}.

\begin{table}[htpb]
    \centering
    \begin{tabular}{p{3.2cm}p{3.2cm}p{3.2cm}p{3.2cm}} \toprule
        & \multicolumn{2}{c}{\textbf{Finite Thrust}} & \textbf{Infinite Thrust} \\ \midrule
        & Constant Specific Impulse & Variable Specific Impulse & Impulsive \\ \midrule
        Description & Thrust limited, usually chemical engine & Power limited, usually electric engine & Arbitrary \(\Delta V\) applied instantly \\
        Objective Function corresponding to~\eqref{eq:max_m} & \(\min \int_{t_0}^{t_f} \Gamma(t) dt\) & \(\min \int_{t_0}^{t_f} \Gamma^2(t) dt\) & \(\min \sum_{i=1}^N \Delta \vec{V}_i\) \\ \bottomrule
    \end{tabular}
    \caption{Propulsion models and objective functions}\label{tab:prop_models}
\end{table}

A priori, all combinations of maneuver x propulsion model seem to be reasonable cases of study. It is unknown whether using the objetive functions from the table or using the generic~\eqref{eq:max_m} is better for the solvers.

\newpage
\section{Propagation model} %decouple model & parameterization

Several orbital dynamics models are available. Naturally the more complex the model, the more precise it is. It is however necessary to have simplified synthesis models for control optimization. Table~\ref{tab:orb_models} discusses some filtered options. For example, there exists a 3D Keplerian Analytical propagator, but its pratical interest is small due to J2, Sun, Moon. Similarly, classical orbital elements are ignored due to the usage of circular/elliptical orbits in the Maneuvers of Interest.

\begin{table}[htbp]
    \centering
    \begin{tabular}{>{\centering\arraybackslash}m{5cm}>{\centering\arraybackslash}m{5cm}>{\centering\arraybackslash}m{5cm}}\toprule
        \textbf{Model} & \textbf{Description} & \textbf{Comments} \\ \midrule
        2D Kepler Analytical & Coasting (zero thrust) model with explicit \(M(t)\) equation & Only suited for coplanar, equatorial(? no J2 influence), short duration (drag), impulsive cases. Available in SatelliteToolbox. \\ \midrule
        3D J2 Osculating Analytical & Coasting model with explicit \(\Theta(t)\) for classical orbital parameters & Suited for LEO 3D, available in SatelliteToolbox \\ \midrule
        2D/3D Cartesian Kepler dynamics & Dynamical equations for \(r\) and \(v\) in continuous time & Suited for all thrust models, equatorial/ coplanar cases. Easy to develop \\ \midrule
        3D Cartesian J2 dynamics & Dynamical equations including J2 perturbation & All thrust models, easy to develop \\ \midrule
        Equinoctial J2 dynamics & Dynamical equations parameterized by equinoctial elements & Might have better numerical properties than Cartesian \\ \midrule
        2D (3D?) Cartesian Moon (+sun?) dynamics & Dynamical equations considering third body influences & Suited for high altitude orbits, components supplied by SatelliteToolbox. \\ \midrule
        Full dynamical model & High fidelity gravity model with Sun and Moon influence, atmospheric drag & Components available in SatelliteToolbox \\ \bottomrule
    \end{tabular}
    \caption{Orbit propagation models}\label{tab:orb_models}
\end{table}

\section{Solvers}

Literature says using indirect methods is inefficient due to high variable count~\cite{Conway_2010}. Therefore necessary conditions are to be used as validation methods, not as solution methods. This leaves direct methods and the experimental sum-of-squares decomposition optimal control~\cite{henrion_occ_meas}. Direct methods are many and vary in precision and computation time. It is not the goal of this work to compare different numerical schemes, suffice to find one capable of satisfying optimality necessary conditions. 

\begin{table}[htbp]
    \centering
    \begin{tabular}{>{\centering\arraybackslash}m{5cm}>{\centering\arraybackslash}m{5cm}>{\centering\arraybackslash}m{5cm}}\toprule
        \textbf{Solver} & \textbf{Description} & \textbf{Comments} \\ \midrule
        Parameter optimization (IPOPT, Newton, \dots) & Optimize finite parameter vector subject to box constraints & Requires parameterizing infinite dimensional control \(T(t)\) by finitely many parameters such as switching times, \(\Delta V\) magnitude, etc \\ \midrule
        Hermite polynomial collocation & Approximate solution \(\Theta(t)\) by cubic segments. Continuity conditions at segment borders enforce dynamics. & Transforms problem into parameter optimization problem with very high number of variables \\ \midrule
        Runge-Kutta parallel shooting & Apply segments of RK integration as variable constraints \(\Theta_{k+1} = RK(\Theta_k)\) & Less variables than collocation, allows for fast-changing controls and slow changing states. Becomes parameter optimization problem \\ \midrule
        Sum-of-Squares decomposition & Continuous-time optimal control method, requires polynomial dynamics and constraints. & Experimental method rarely applied; would be innovative. Requires variable transformations to make dynamics polynomial; might work with Cartesian models, MAYBE equinoctial as well. \\ \bottomrule
    \end{tabular}
    \caption{Solver options}\label{tab:solvers}
\end{table}

\section{Planning}

\begin{enumerate}
    \item Solve LEO maintenance, impulsive propulsion, 2D Kepler analytical, parameter optimization
    \item Solve LEO maintenance, impulsive propulsion, 3D J2 analytical, parameter optimization
    \item Solve LEO maintenance, CSI propulsion, 2D Kepler dynamics, RK
    \item Solve LEO maintenance, CSI propulsion, 3D J2 dynamics, RK
    \item Solve LEO maintenance, VSI propulsion, 3D J2 dynamics, RK
    \item Solve Rideshare Phasing, 3 propulsion types, 3D J2 dynamics, RK
    \item Solve LEO maintenance, propulsion?, 2D Kepler dynamics, SoS
    \item Solve other cases, CSI propulsion, 3D J2 dynamics, RK
    \item ????
\end{enumerate}

\section{Comments}

\begin{itemize}
    \item Too many combinations; automation is possible but not all are interesting.
    \item Really want to test SoS decomposition but it can't be applied to all models.
    \item Use simplest synthesis model that satifies necessary conditions in full dynamical model?
    \item Instead of doing all combinations propulsion model x propagation, pick real spacecraft as case studies?
    \item At least one maneuver should be solved by many combinations of propulsion model, propagation model, solver.
    \item Recover some analytical results such as Hohmann, perigee burn increases energy more efficiently, result from \cite{sandro_quasi_circ}?
\end{itemize}